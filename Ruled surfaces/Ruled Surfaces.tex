\documentclass[12pt]{article}
\usepackage{amsmath}
\usepackage{enumerate}
\usepackage{mathrsfs} 
\usepackage{amsthm}
\usepackage{amsfonts}
\usepackage{amssymb}
\usepackage{latexsym} 
%\usepackage{epsfig}
%\usepackage{graphicx}
%\usepackage[dvips]{graphicx}
\usepackage{tikz}
\usepackage{tikz-cd}



\usepackage[matrix,tips,graph,curve]{xy}

\newcommand{\mnote}[1]{${}^*$\marginpar{\footnotesize ${}^*$#1}}
\linespread{1.065}

\makeatletter

\setlength\@tempdima  {5.5in}
\addtolength\@tempdima {-\textwidth}
\addtolength\hoffset{-0.5\@tempdima}
\setlength{\textwidth}{5.5in}
\setlength{\textheight}{8.75in}
\addtolength\voffset{-0.625in}

\makeatother

\makeatletter 
\@addtoreset{equation}{section}
\makeatother


\renewcommand{\theequation}{\thesection.\arabic{equation}}

\theoremstyle{plain}
\newtheorem{theorem}[equation]{Theorem}
\newtheorem{corollary}[equation]{Corollary}
\newtheorem{conjecture}[equation]{Conjecture}
\newtheorem{lemma}[equation]{Lemma}
\newtheorem{proposition}[equation]{Proposition}
\theoremstyle{definition}
\newtheorem{definition}[equation]{Definition}
\newtheorem{definitions}[equation]{Definitions}
%\theoremstyle{remark}

\newtheorem{remark}[equation]{Remark}
\newtheorem{remarks}[equation]{Remarks}
\newtheorem{exercise}[equation]{Exercise}
\newtheorem{example}[equation]{Example}
\newtheorem{examples}[equation]{Examples}
\newtheorem{notation}[equation]{Notation}
\newtheorem{question}[equation]{Question}
\newtheorem{assumption}[equation]{Assumption}
\newtheorem*{claim}{Claim}
\newtheorem{answer}[equation]{Answer}
%%%%%% letters %%%%

\newcommand{\fa}{\mathfrak{a}}
\newcommand{\fb}{\mathfrak{b}}
\newcommand{\fd}{\mathfrak{d}}
\newcommand{\fm}{\mathfrak{m}}
\newcommand{\fp}{\mathfrak{p}}
\newcommand{\fq}{\mathfrak{q}}

\newcommand{\IA}{\mathbb{A}}
\newcommand{\IN}{\mathbb{N}}
\newcommand{\IF}{\mathbb{F}}
\newcommand{\IP}{\mathbb{P}}
\newcommand{\IZ}{\mathbb{Z}}

\newcommand{\sD}{\mathcal{D}}
\newcommand{\sI}{\mathcal{I}}
\newcommand{\sO}{\mathcal{O}}
\newcommand{\sP}{\mathcal{P}}
\newcommand{\sQ}{\mathcal{Q}}
\newcommand{\sT}{\mathcal{T}}
\newcommand{\sU}{\mathcal{U}}

\newcommand{\shF}{\mathscr{F}}
\newcommand{\shG}{\mathscr{G}}
\newcommand{\shI}{\mathscr{I}}
%%%%%%% macros %%%%%

%% my definitions %%%

\newcommand{\End}{\mathrm{End}}
\newcommand{\tr}{\mathrm{tr}}
\newcommand{\Hom}{\mathrm{Hom}}
\newcommand{\Aut}{\mathrm{Aut}}
\newcommand{\Trace}{\mathrm{Trace}\,}
\newcommand{\rank}{\mathrm{rank}}
\renewcommand{\deg}{\mathrm{deg}\,}
\newcommand{\Spec}{\rm Spec\,}
\newcommand{\Proj}{\rm Proj\,}
\newcommand{\Sym}{\mathrm{Sym \,}}
\newcommand{\Span}{\mathrm{Span \,}}
\renewcommand\dim{{\rm dim\,}}
\newcommand{\codim}{{\rm codim\,}}
\renewcommand\det{{\rm det\,}}
\newcommand{\im}{{\rm Im\,}}


\newcommand\iso{{\, \simeq \,}} 
\newcommand\tensor{{\otimes}}
\newcommand\Tensor{{\bigotimes}} 
\newcommand\union{\bigcup} 
\newcommand\onehalf{\frac{1}{2}}
\newcommand\trivial{{\mathbb I}}
\newcommand\wb{\overline}

%%%%%Delimiters%%%%

\newcommand{\<}{\langle}
\renewcommand{\>}{\rangle}

%\renewcommand{\(}{\left(}
%\renewcommand{\)}{\right)}


%%%% Different kind of derivatives %%%%%

\newcommand{\delbar}{\bar{\partial}}
\newcommand{\pdu}{\frac{\partial}{\partial u}}
%\newcommand{\pd}[1][2]{\frac{\partial #1}{\partial #2}}

%%%%% Arrows %%%%%
\newcommand{\induce}{\rightsquigarrow}
\newcommand{\into}{\hookrightarrow}
\newcommand{\onto}{\twoheadrightarrow}
\newcommand{\tto}{\longmapsto}
\def\llra{\longleftrightarrow}
\def\wt{\widetilde}
\def\wtilde{\widetilde}
\def\what{\widehat}
\def\bf{\textbf}
\def\it{\textit}
%%%%%%%%%%%%%%%%%%% Ziquan's definitions %%%%%%%%%%%%%%%%%%%%
\newcommand{\Ann}{\mathrm{Ann}}
\newcommand{\height}{\mathrm{height \,}}
\newcommand{\Div}{\mathrm{Div}}
\newcommand{\sE}{\mathcal{E}}
\newcommand{\p}{\partial}
\newcommand{\Ohm}{\Omega}
\newcommand{\w}{\omega}
\newcommand{\sing}{\mathrm{sing}}
%%%%%%%%%%%%% new definitions for the positive mass paper %%%%%%%%%

\newcommand{\sperp}{{\scriptscriptstyle \perp}}
\newcommand{\Qmed}{\mathcal{Q}_r^\mathrm{medium}}
\newcommand{\Qhigh}{\mathcal{Q}^\mathrm{high}}
\newcommand{\uppermu}{\overline{\mu}}
\newcommand{\lowermu}{\underline{\mu}}
\newcommand{\res}{\mathrm{res}}
\newcommand{\ev}{\mathrm{ev}}
\newcommand{\pr}{\mathrm{pr}}
\newcommand{\Prob}{\mathrm{Prob}}
\newcommand{\bq}{\mathbf{q}}
%%%%%%%%%%%%%%%%%%%%%%%

%%%%%%%%%%%%%%%%%%%%%%%%%%%%%%%%%%%%%%%%%%%%%



%
\begin{document}
%

\title{On the Density of Simply Ramified Curves in Ruled Surfaces}
\author{Ziquan Yang}


\date{\today}

\maketitle
\tableofcontents

 
%\setcounter{secnumdepth}{1} 

\setcounter{section}{0}

\section{Introduction}
In \cite{Wood}, Erman and Wood generalized Poonen's Bertini smoothness theorem over a finite field to semiample divisors and determined the density of smooth curves in Hirzebruch surfaces. More precisely, they proved the following: 

\begin{theorem}
For fixed $n \ge 3$, $d \to \infty$, the probability that a curve of bidegree $(n, d)$ in a Hirzebruch surface $X$ is smooth is 
$$ \prod_{P \in \IP^1_{\IF_q}} (1 - q^{ - 2 \deg P})(1 - q^{-3 \deg P}) = \zeta_{P \in \IP^1_{\IF_q}}(2)^{-1} \zeta_{P \in \IP^1_{\IF_q}}(3)^{-1} $$
\end{theorem}
Hirzebruch surfaces are $\IP^1$-bundles over $\IP^1$, which are classified by non-negative integers. As we can see, the probability in the above result is independent of the global twisting of the Hirzebruch surface. In fact, heuristically what Erman and Wood proved was that smoothness is independent across fibers, though not necessarily within a fiber. In particular, this implies that the global twisting of the Hirzebruch surface as a $\IP^1$-bundle over $\IP^1$ does not come into play.  

Hirzebruch surfaces are special cases of ruled surfaces. We use the following definition of a ruled surface, which can be found in \cite{Hart}:

\begin{theorem}
\label{main}
A ruled surface is a surface $X$, together with a surjective morphism $\pi : X \to C$ to a (nonsingular) curve $C$, such that the fiber $X_P$ is isomorphic to $\IP^1_{\kappa(P)}$ for every point $P \in C$, and such that $\pi$ admits a section, i.e. a morphism $\sigma : C \to X$ such that $\pi \circ \sigma$ is identity on $C$. 
\end{theorem}

Through previous work we extended the techniques developed in \cite{Poonen} and \cite{Wood} to compute the asympotic density as $d \to \infty$ of simply ramified curves of bidegree $(3, d)$ in $\IP^1 \times \IP^1$ when $n = 3$. Now we want to extend the result to all ruled surfaces over a projective curve $C$. 

We first explain some notations and setups. We assume that $C$ is embedded as a quasiprojective curve in some $\IP^M$ over a finite field $\IF_q$ with characteristic $p$.Let $A$ be the divisor corresponding to $\sigma(C)$ and let $E$ be the divisor corresponding to the fiber $X_{P_0}$ for some $P_0 \in C$ with $\deg P_0 \ge 2g(C)$ such that the projection $\pi : X \to C$ is induced by the linear system of $E = \pi^{-1}P_0$. Let $R_{3, d} = H^0(X, \sO(3A + dE))$. If $f \in R_{3, d}$ is a nonzero section, we use $H_f$ to denote the zero locus of $f$, which is a curve in $X$. Given $\sP \subseteq \cup_d R_{3, d}$ a subset, we define the probability $\Prob(f \in \sP)$ by 
$$ \Prob(f \in \sP) := \lim_{d \to \infty} \frac{|\sP \cap R_{3, d}|}{|R_{3, d}|} $$
if the limit exists. 
Recall that a curve $D \in X$ is called \textit{simply ramified} if it is smooth, and for each point $Q \in D$, the ramification index (with respect to $\pi$) $e_Q(D) \le 2$. Since we are going to deal with many curves, we add $D$ as a parameter in the notation. We will say that $f$, or $H_f$ is ``good" at a point $Q$ if $H_f$ is smooth at $Q$ and $e_Q(H_f) \le 2$ and ``bad" otherwise. The convention is, if $Q \not\in H_f$, then both requirements are automatically satisfied. 
We will prove the following result: 

\begin{theorem}
Let $R_{3, d}$ and $\pi$ be defined as above. Define $\sD \subseteq \cup_d R_{3, d}$ to be the subset consisting of those sections $f$ such that $H_f$ is simply ramified with respect to $\pi$. Then 
$$ \Prob( f \in \sD) = \zeta_C(2)^{-2} $$
\end{theorem}
We will bound the probability of being bad at some point $Q$ when $\pi(Q)$ has a low, medium and high degree respectively. Therefore for some $e_0 \ge 1$ we define: 
\begin{align*}
\sP_{e_0}^{\mathrm{low}} &= \union_{d \ge 0} \{ f \in R_{3, d} : f \textit{ is good at all $Q$}, \deg \pi(Q) < e_0\}\\
\sQ_{e_0}^{\mathrm{med}} &= \union_{d \ge 0} \{f \in R_{3, d} : f \textit{ is bad at some $Q$, }\deg \pi(Q) \in [e_0, \lfloor d/p \rfloor]\}\\
\sQ^{\mathrm{high}} &= \union_{d \ge 0} \{f \in R_{3, d} : f \textit{ is bad at some $Q$, }\deg \pi(Q) > d/p\}
\end{align*}

\section{Lemmas and proofs}

\subsection{Points of low degree}
\begin{lemma}
If $\fd$ is a divisor on $C$, then there is a constant $m$ such that 
$$ H^1(X, \sO_X(3A) \tensor \pi^* \sO_C(\fd) ) = 0 $$ whenever $\deg(\fd) \ge m$. 
\end{lemma}
\begin{proof}
Use the exact sequence 
\begin{align*}
0 \to H^1(C, \pi_* \sO_X(3A) \tensor \sO_C(\fd)) \to  & H^1(X, \sO_X(3A) \tensor \pi^* \sO_C(\fd))  \\ 
& \to H^0(C, R^1 \pi_* \sO_X(3A) \tensor \sO_C(\fd))
\end{align*}
$H^0(C, R^1 \pi_* \sO_X(3A) \tensor \sO_C(\fd)) = 0$ since $H^1(\IP^1, \sO(3)) = 0$. When $m$ is sufficiently large, $H^1(C, \pi_* \sO_X(3A) \tensor \sO_C(\fd)) = 0$ for $\deg(\fd) \ge m$ by Serre's vanishing theorem. 
\end{proof}


\begin{lemma}
\label{Serre}
There exists a constant $m$ for $C$ such that whenever $X_W$ is the fiber over a zero dimensional subscheme $W \subseteq C$, the map
$$ H^0(X, \sO_X(3A + dE)) \to H^0(X_W, \sO_{X_W}) $$
is surjective for each $d \ge m + \deg W$. 
\end{lemma}
\begin{proof}
$W$ gives a divisor on $C$ and $X_W$ gives a divisor on $X$. The exact sequence 
$$ 0 \to \shI_{X_W} \tensor \sO_X(3A + dE) \to \sO_X(3A + dE) \to \sO_X(3A + dE)|_{X_W} \to 0$$
gives a long exact cohomology sequence 
\begin{align*} H^0(X, \sO_X(3A + dE)) \to H^0(X_W, &\sO_X(3A + dE)|_{X_W}) \\ &\to H^1(X, \sO_X(3A + dE - X_W)) \to \cdots 
\end{align*}
By the previous lemma, $H^1(X, \sO_X(3A + dE - X_W)) = 0$ when $d \ge m + \deg W$ for some constant $m$.
\end{proof}

Now we want to do some local analysis on the fiber. Let $P \in C$ be a point of degree $e$ contained in some affine chart $\Spec B \subseteq C$ where $ B = \IF_q[t_1, \cdots, t_M]/I$ for some ideal $I$. Let $\fm \subseteq B$ be the maximal ideal corresponding to $P$. Then $P^{(2)} = \Spec B/\fm^2$ as a closed subscheme of $C$ and $X_P = \IP^1_{B/\fm^2}$. Denote the reduction map $B/\fm^2 \to B/\fm$ by $g \mapsto \overline{g}$. We extend it to a map $H^0(\IP^1_{B/\fm^2}, \sO(3)) \to H^0(\IP^1_{B/\fm}, \sO(3))$ by applying $B/\fm^2 \to B/\fm$ to each coefficients. Let $\varphi_P : R_{3, d} \to H^0(\IP^1_{B/\fm^2}, \sO(3))$ be the restriction map. $H_f$ is smooth at a point $Q \in \pi^{-1}(P)$ if and only if $\varphi_P(f)$ does not vanish at $Q^{(2)} \in \IP^1_{B/\fm^2}$, and $e_Q(H_f) \ge 3$ if and only if $\wb{\varphi_P(f)}$ does not have a zero of multiplicity $\ge 3$ at $Q$ on $\IP^1_{B/\fm}$. In other words, we can determine if $H_f$ is good at all points on $\pi^{-1}(P)$ by looking at the image of $f$ in $H^0(\IP^1_{B/\fm^2}, \sO(3))$, so we observe that:

\begin{lemma}
\label{Low}
$$\Prob(f \in \sP_{e_0}^{\mathrm{low}})  = \prod_{\deg(P) < e_0}  \Prob( H_f \textit{ is good at all points } Q \in H_f \cap X_P ) $$
\end{lemma}
\begin{proof}
Let $P_1, P_2, \cdots, P_s \in C$ be the points with degree $< e_0$. Apply Lemma~\ref{Serre} $W = \coprod_i P_i^{(2)}$. 
\end{proof}

We make the following convention:
 We call a pair $(F_1, F_2) \in H^0(\IP^1_{A/\fm}, \sO(3))^2$ ``bad" if it is one of the following 3 types:
\begin{enumerate}
\item $F_1$ has a root of multiplicity $\ge 2$ at a point where $F_2$ also vanishes. 
\item $F_1$ has a root of multiplicity $3$ at a point where $F_2$ does not vanish. 
\item $F_1 \equiv 0$.
\end{enumerate}

Since $C$ is some smooth curve, $\sO_{P, C}$ is a discrete valuation ring and we can find a uniformizer $u$ such that $(u) = \fm \sO_{P, C}$.Let $\, \wt{} : B/\fm \to B/\fm^2$ be a $\IF_q$-linear map that is a section to the reduction map $\, \bar{} : B/\fm^2 \to B/\fm$. We extend it to a map $H^0(\IP^1_{B/\fm^2}, \sO(3)) \to H^0(\IP^1_{B/\fm}, \sO(3))$ in the same way we extended the reduction map. Then we have the following:
\begin{lemma}
$$ H^0(\IP^1_{B/\fm}, \sO(3))^2 \to H^0(\IP^1_{B/\fm^2}, \sO(3)) $$
given by $(F_1, F_2) \mapsto \wt{F}_1 + uF_2$ is a bijection. Moreover, $f \in R_{3, d}$ is ``good" at all points $Q \in \pi^{-1}(P)$ if and only if $\varphi_P(f)$ does not correspond to a bad pair. 
\end{lemma}
\begin{proof}
This is essentially a version of Lemma 9.7 in \cite{Wood} tailored to our situation. The map $f \mapsto ( \wb{f}, (f - \wt{\wb{f}})/u)$ is an inverse to the map given in the lemma. Let $(F_1, F_2)$ be the pair corresponding to $\varphi_P(f)$. $\varphi_P(f)$ vanishes on $Q^{(2)}$, i.e. $H_f$ is singular at $Q$, if and only if $Q$ is a double root of $F_1$ and a root to $F_2$. Similarly, $e_Q(H_f) \ge 3$ if and only if $Q$ is root to $F_1$ of multiplicity $\ge 3$. 
\end{proof}

\begin{lemma}
\label{count}
The density of good pairs in $H^0(\IP^1_{\kappa(P)}, \sO(3))^2$ is 
$$ (1 - q^{-2e})^{2} $$
\end{lemma}
\begin{proof}
To simplify notation, we can replace $q^e$ by $\bq$ in the proof. We first count type 1 pairs. Let $P$ be the point that is a double root to $F_1$ and a root to $F_2$. Note that $\deg P = 1$. $F_1$ is fixed up to rescaling by $\IF_\bq^*$ after we choose a third root, which can be any point of degree $1$. Therefore there are $(\bq + 1)(\bq - 1)$ choices for $F_1$. The probability that $F_2$ vanishes at $P$ is $\bq^{-1}$, so we have $q^3$ choices for $F_2$. Since we have $(\bq + 1)$ choices for $P$, there are $\bq^3(\bq + 1)^2(\bq - 1)$ type 1 pairs. 

To count type 2 pairs, let $P$ be the triple root to $F_1$. Again there are $\bq + 1$ choices. At each $P$, there are $(\bq-1)$ homogeneous polynomials of degree $3$ which have $P$ as a triple root, since the leading coefficient will uniquely determine such a polynomial. We have $\bq^4 - \bq^3$ choices for $F_2$. In total there are $ (\bq + 1)(\bq - 1)(\bq^4 - \bq^3)$
pairs of type 2. 

Finally when $F_1 = 0$, $F_2$ can be anything, so we have $\bq^4$ type 3 pairs.
Therefore the density of good pairs is 
$$ 1 - \bq^{-8}(\bq^3(\bq + 1)^2(\bq - 1) + (\bq + 1)(\bq - 1)(\bq^4 - \bq^3) + \bq^4) = (1 - \bq^{-2})^2 $$ 
\end{proof}

\subsection{Points of medium degree}
\begin{lemma}
\label{Medium}
$$\lim_{e_0 \to \infty} \Prob( f \in  \sQ_{e_0}^{\mathrm{med}}) = 0 $$
\end{lemma}
\begin{proof}
Let $P$ be a point of degree $e < \lfloor d/p \rfloor$ on $C$ and let $B_P$ denote the event that a randomly chosen $f \in R_{3, d}$ is ``bad" at some point in the fiber $X_P$. By Lemma~\ref{Serre} applied to $W = P^{(2)}$, there is some constant $m$ such that when $d \ge m + 2 \deg P$, $R_{3, d} \to H^0(X_W, \sO_{X_W})$ is surjective. Since we assumed $p \neq 2$ and $\deg P < \lfloor d/p \rfloor$, when $d$ is large enough, we automatically have that $d \ge m + 2 \deg P$. With this surjectivity, we can infer from Lemma~\ref{count} that the probability of event $B_P$ is bounded by $q^{-2e}$. Of course, the probability of $\cup_P B_P$ is bounded by the sum of these individual probabilities, so we have:
\begin{align*}
\Prob(f \in R_{3, d} \cap \sQ_{e_0}^{\mathrm{med}}) &\le \sum_{e = e_0}^{\lfloor d/p \rfloor} (\text{number of points of degree $e$ in $C$})q^{-2e} \\
&\le \sum_{e = e_0}^{\lfloor d/p \rfloor}|C(\IF_{q^e})| q^{-2e} \\
&\le \sum_{e = e_0}^{\infty} c q^{e} q^{-2e} \\
&= \frac{c q^{-e_0}}{1 - q^{-1}}
\end{align*}
The third inequality is given by Lang-Weil bound, where $c$ is a constant depending only on $C$. 
\end{proof}

\subsection{Points of high degree} 
The following is a simple linear-algebraic observation. It is used in \cite{Wood} without explicit explanation. For the reader's convenience, we single it out since it justifies a main step in simplifying the proof of Lemma~\ref{Decoup} later. 
\begin{lemma}
\label{linear}
Let $V, W$ be finite dimensional vector spaces over $\IF_q$ and let $V_0 \subseteq V$ be a subspace. Let $p : W \to V$ be a linear transformation, then 
$$ \frac{|p^{-1} V_0|}{|W|} \ge \frac{|V_0|}{|V|}$$
\end{lemma}
\begin{proof}
Let $I$ be the image of $p$. We are to verify 
$$ |V||V_0 \cap I| \ge |V_0||I| $$
Take $\log_p$ on both sides to give
$$ \dim V  + \dim V_0 \cap I \ge \dim V_0 + \dim I$$
which is true since 
$$ \dim V_0 \cap I \ge \dim V_0 + \dim I - \dim V $$  
\end{proof}

\begin{lemma}
\label{book-keeping}
Let $B = \IF_q[x_1, \cdots, x_N]/I$ be a ring such that $U = \Spec B$ is a smooth quasi-projective scheme of dimension $m$. Let $\Ohm^1_{U/\IF_q}$ be the sheaf of differentials of $U$. Suppose $\Ohm^1_{U/\IF_q} = \oplus_{i = 1}^m \sO_U ds_i$ for some $s_i$'s, $s_i \in B$. Then there are derivations $D_i : B \to B$ such that for each $f \in B$, $df = \sum_{i = 1}^m D_i f ds_i$. 
\end{lemma}
\begin{proof}
For each $f$, there are unique $a_i$'s such that $df = a_1 ds_1 + a_2 ds_2$ since $\Ohm^1_{U/\IF_q}(U) = \oplus_{i = 1}^N B ds_i$. The map $f \mapsto a_i$ defines $D_i$. The main point is that $D_i$'s are derivations. That $D_i$'s are additive and $D_i a = 0$ for any $a \in \IF_q$ are clear. We only need to show that it satisfies the Leibniz rule. Note that if $f, g \in B$, then 
$$ d(fg) = f dg + g df = f(\sum_{i = 1}^N g ds_i) + g(\sum_{i = 1}^N f ds_i) = \sum_{i = 1}^N (f D_i g + g D_i f) ds_i $$
Uniqueness of such a decomposition forces $D_i(fg) = fD_i g + g D_i f$.  
\end{proof}

The following lemma is part of Lemma 5.3 in \cite{Wood}. For completeness we reproduce the statement below: 
\begin{lemma}
\label{5.3}
Let $W \subseteq \IA^N \times \IA^M$ be a closed subscheme such that $\pi_1$ is an isomorphism on $W$ and $\pi_2$ is supported at a closed point $P$ of degree $e$. Let $r = \deg(\pi_2(W))$. Let $S_{n, d} \subset \IF_q[s_1, \cdots, s_N, t_1, \cdots, t_M]$ be the set of polynomials of degree at most $n$ in $s_i$'s and at most $d$ in $t_j$'s. Consider the restriction map 
$$ \phi_{W, n, d} : S_{n, d} \to H^0(W, \sO_W) $$
Then for $n, d \ge 0$, we have $|\im(\phi_{W, n, d})| \ge q^{\min(d+1, r)}$. 
\end{lemma}
\begin{remark}
In the above lemma, we may replace $W$ by a dense open subset $W'$, since there is a natural inclusion $H^0(W', \sO_W) \subseteq H^0(W, \sO_W)$. Therefore the assumption that $W$ is closed is not necessary. 
\end{remark}


\begin{lemma}
\label{Decoup}
When $n_0$ sufficiently large, we have that $A' = nA + n_0 E$ is a very ample divisor. Let $f \in R_{3, d+n_0}$ be chosen and random and $j \in \IN$ be fixed. The probability $P^*$ that $H_f$ contains a point $Q \in X$ with $\deg \pi(Q) \ge j$, $e_Q(H_f) \ge 3$ is at most
$$ O((3 + d)q^{- \min\{ \lfloor d/p \rfloor + 1, j    \}})$$
\end{lemma}
\begin{proof}
We call a point $Q \in X$ \textit{admissible} if $\deg \pi(Q) \ge j$. 


On $C$ we can find trivializing charts. That is, $C$ can be covered by open affine sets $V_\alpha = \Spec B_\alpha$, such that $\pi^{-1}(V_\alpha) \subseteq X$ is isomorphic to $V_\alpha \times \IP^1$, and $\pi$ resticted to $\pi^{-1}(V_\alpha) \iso V_\alpha \times \IP^1$ is the projection map $V_\alpha \times \IP^1 \to V_\alpha$. 
$V_\alpha \times \IP^1$ is covered by two copies of $V_\alpha \times \IA^1$.  Suppose $V_\alpha = \Spec \IF_q[v_1, \cdots, v_{M_\alpha}]/I$ and $\IA^1 = \Spec \IF_q[y]$. $V_\alpha$ can be viewed as an affine curve in $\IA^{M_\alpha}$. Without loss of generality, we may further restrict $V_\alpha$ so that some $d v_1$ generates the module of differentials. Therefore the differentials on $V_\alpha \times \IA^1$ are generated by (more precisely, pullbacks of) $dv_1$ and $dy$. By Lemma~\ref{book-keeping}, we obtain derivations $D_{v_1}, D_y : H^0(V_\alpha \times \IA^1, \sO) \to H^0(V_\alpha \times \IA^1, \sO)$. Suppose $H_f$ is a curve in $V_\alpha \times \IA^1$ is defined by $f \in H^0(V_\alpha \times \IA^1, \sO)$. Then for a closed point $Q \in V_\alpha \times \IA^1$, $e_{Q}(H_f) \ge 3$ if and only if $f = D_y f = D_y^2 f = 0$ at $Q$. 

Let $\iota : X \to \IP^N$ be the embedding induced by $A'$. We can cover $X$ with open subsets and show the conclusion holds for each of them. This suffices since $X$ is quasi-compact.

We may first reduce to an open subset $U \subseteq X \cap \IA^N$ that is contained in some $V_\alpha \times \IA^1$ described above. Let $\IA^M = \Spec \IF_q[t_1, \cdots, t_M] \subseteq \IP^M$ be an affine chart. Replace $U$ by $U \cap \pi^{-1}(\IA^M)$ if needed, we may assume that $\pi(U) \subseteq \IA^M$.

Next let $\IA^N = \Spec \IF_q[s_1, \cdots, s_N] \subseteq \IP^N$. By again reducing to a finite cover and by possibly relabeling the $s_i$'s, we can assume that $ds_1, ds_2$ generate the differentials on $U$. By Lemma~\ref{book-keeping}, we obtain derivations $D_i : \sO_X(U) \to \sO_X(U)$ such that on $U$, $df = D_1 f ds_1 + D_2 f ds_2$. However, since $d v_1, dy$ also generate the differentials on $U$, there are regular functions $b_{i,j} \in \sO_X(U)$ such that $ds_1 = b_{1, 1} dv_1 + b_{1, 2} dy$ and $ds_2 = b_{2, 1} dv_1 + b_{2, 2} dy$. Therefore we observe that
$$ D_y = b_{1, 2} D_1 + b_{2,2} D_2$$
Let the notation $S_{n, d}$ be defined as in Lemma~\ref{5.3}. On $\IA^N \times \IA^M$ we may naturally identify $S_{3, d}$ with $H^0(\IP^N \times \IP^M, \sO(3, d))$. 

For any $Q \in U$, we have maps
$$ S_{3, d} \iso H^0(\IP^N \times \IP^M, \sO(3, d)) \to R_{3, d + n_0} \to \sO_X(U) \to H^0(Q^{(3)}, \sO(3, d)_{Q^{(3)}}) $$ 
We denote the composite map $S_{3, d} \to \sO_X(U)$ by $\phi$. By Lemma~\ref{linear}, it is at least as likely that $H_{\phi(g)} \cap U$ ramifies thrice at $Q$ for a randomly chosen element $g \in S_{3, d}$ as it is that $H_f$ ramifies thrice at $Q$ for a randomly chosen $f \in R_{3, d}$. Thus we reduce to studying a randomly chosen $g \in S_{3, d}$. 

We apply the Poonen's decoupling idea to decouple the first and second order partial derivatives in this context. To choose $g \in S_{3, d}$ at random we choose $g_0 \in S_{3, d}$, $g_1, g_2, h \in S_{0, \lfloor d/p \rfloor} $ at random and put 
$$ g = g_0 + g_{2}^p s_1^2 + g^p_{1} s_1 + h^p $$
Now $D_y \phi(g) = 0$ amounts to 
$$b_{1, 2} ( D_1 \phi(g_0) + 2 \phi(g_{2})^p s_1 + \phi(g_{1})^p) + b_{2, 2}(D_2 \phi(g_0)) = 0 $$
And $D_y^2 \phi(g) = 0$ amounts to: 
$$ b_{1, 2}^2 (D_1^2 \phi(g_0) + 2 \phi(g_{2})^p) + b_{1, 2}b_{2, 2} (D_1 D_2 \phi(g_0)) + b_{2, 2}^2(D_2^2 \phi(g_0)) = 0 $$


Define $W_2 = U \cap \{ D_y^2 \phi(g) = 0 \}$, $W_1 = U \cap \{D_y^2 \phi(g) = D_y \phi(g) = 0\}$ and $W_0 = U \cap \{ D_y^2 \phi(g) = D_y \phi(g) = \phi(g) = 0\}$. \\\\
\textit{Claim: }Conditioned on a choice of $g_0$, the probability that $\dim W_2 = 1$ and $W_2$ contains an admissible point $Q$ is $$1 - \sO((n + d)q^{- \min\{ \lfloor d/p \rfloor + 1, j\}})$$ Let $V_1, \cdots, V_c$ be irreducible components of $\{D_y^2(\phi(g_0)) = 0\}_{\mathrm{red}}$.    
For each $V_k$ that contains an admissible point, those $g_2$ that will make $D_y^2 \phi(g)$ vanish identically on $V_k$ form a coset of the linear map 

$$ \varphi_k : S_{0, \lfloor d/p \rfloor} \to H^0(V_k, \sO_{V_k})$$ 

A lower bound on $|\im \varphi_k|$ gives an upper bound for the probability that $D_y^2(g)$ vanishes on $V_k$, which is simply the inverse of $|\im \varphi_k|$. If $\dim \pi(V_k) \ge 1$, we see that the pullback of some $t_i$, and hence any polynomial in $t_i$, does not vanish on $V_k$. In this case, we have $|\im \varphi_k| \ge q^{\lfloor d/p \rfloor + 1}$. If $\dim \pi(V_k) = 0$, we invoke Lemma~\ref{5.3} to obtain $|\im \varphi_k| \ge q^{\min \{\lfloor d/p \rfloor + 1, j \}  }$. The claim hence follows. 

Now we may randomly choose $g_1$ to fix $W_2$ and then choose $h$ to fix $W_0$. Similarly, we can show that the probability that $\dim W_1 = 0$ conditioned on a choice of $g_0, g_2$ such that $\dim W_2 = 1$, and the probability that $W_0$ contains no admissible points conditioned on a choice of $g_1$ such that $\dim W_1 = 0$ are both bounded by $$1 - O((3 + d)q^{- \min\{ \lfloor d/p \rfloor + 1, j\}})$$
Therefore we have shown that 
$$ P^* \le 1 - (1 - O((3 + d)q^{- \min\{ \lfloor d/p \rfloor + 1, j\}}))^3 = O((3 + d)q^{- \min\{ \lfloor d/p \rfloor + 1, j\}}) $$  

\end{proof}

\begin{lemma}
\label{High}
$\Prob( f\in \sQ^{\mathrm{high}}) = 0$
\end{lemma}
\begin{proof}
We decompose $\sQ^{\mathrm{high}}$ as $\sQ^{\mathrm{high}}_{\mathrm{sing}} \cup \sQ^{\mathrm{high}}_{\mathrm{multi}}$. As their names suggest, $\sQ^{\mathrm{high}}_{\mathrm{sing}}$ are those curves that are bad for being singular and $\sQ^{\mathrm{high}}_{\mathrm{multi}}$ for having a point with ramification degree $\ge 3$. 
It is already shown by Lemma 5.4 in \cite{Wood} that $$ \Prob(f \in \sQ^{\mathrm{high}}_{\mathrm{sing}}) = 0 $$ and hence we only need to show
$$ \Prob(f \in \sQ^{\mathrm{high}}_{\mathrm{multi}}) = 0 $$
We apply Lemma~\ref{Decoup} with $j = \lfloor d/p \rfloor$. Then take $d \to \infty$.  
\end{proof}

\subsection{Proof of main result}
Now we have all the ingredients to prove Theorem~\ref{main}. For each $e_0$, we have that 
$$  \sP_{e_0}^{\mathrm{low}} \subseteq \sD \subseteq \sP_{e_0}^{\mathrm{low}} \cup \sQ_{e_0}^{\mathrm{med}} \cup \sQ^{\mathrm{high}}$$
Therefore 
$$ \Prob(f \in \sP_{e_0}^{\mathrm{low}}) \le \Prob(f \in \sD) \le \Prob(f \in \sP_{e_0}^{\mathrm{low}} \cup \sQ_{e_0}^{\mathrm{med}} \cup \sQ^{\mathrm{high}})$$
Now take $e_0 \to \infty$, Lemma~\ref{Low}, \ref{count}, \ref{Medium}, \ref{High} combine the give the result. 

\begin{thebibliography}{9} 
\bibitem{Wood}
D. Erman and M.M. Wood, \textit{Semiample Bertini theorems over finite fields}, Duke Mathematical Journal 164(2015), no. 1, 1-38

\bibitem{Poonen}
B. Poonen, \textit{Bertini theorems over finite fields}, Ann. of Math. (2) 160 (2004), no. 3, 1099-1127.


\bibitem{Hart}
R. Hartshorne, \textit{Algebraic geometry}, Springer-Verlag, New York, 1977, Graduate Texts in Mathematics, No. 52

\end{thebibliography}


\end{document}
