\documentclass[12pt]{article}
\usepackage{amsmath}
\usepackage{enumerate}
\usepackage{mathrsfs} 
\usepackage{amsthm}
\usepackage{amsfonts}
\usepackage{amssymb}
\usepackage{latexsym} 
%\usepackage{epsfig}
%\usepackage{graphicx}
%\usepackage[dvips]{graphicx}
\usepackage{tikz}
\usepackage{tikz-cd}



\usepackage[matrix,tips,graph,curve]{xy}

\newcommand{\mnote}[1]{${}^*$\marginpar{\footnotesize ${}^*$#1}}
\linespread{1.065}

\makeatletter

\setlength\@tempdima  {5.5in}
\addtolength\@tempdima {-\textwidth}
\addtolength\hoffset{-0.5\@tempdima}
\setlength{\textwidth}{5.5in}
\setlength{\textheight}{8.75in}
\addtolength\voffset{-0.625in}

\makeatother

\makeatletter 
\@addtoreset{equation}{section}
\makeatother


\renewcommand{\theequation}{\thesection.\arabic{equation}}

\theoremstyle{plain}
\newtheorem{theorem}[equation]{Theorem}
\newtheorem{corollary}[equation]{Corollary}
\newtheorem{conjecture}[equation]{Conjecture}
\newtheorem{lemma}[equation]{Lemma}
\newtheorem{proposition}[equation]{Proposition}
\theoremstyle{definition}
\newtheorem{definition}[equation]{Definition}
\newtheorem{definitions}[equation]{Definitions}
%\theoremstyle{remark}

\newtheorem{remark}[equation]{Remark}
\newtheorem{remarks}[equation]{Remarks}
\newtheorem{exercise}[equation]{Exercise}
\newtheorem{example}[equation]{Example}
\newtheorem{examples}[equation]{Examples}
\newtheorem{notation}[equation]{Notation}
\newtheorem{question}[equation]{Question}
\newtheorem{assumption}[equation]{Assumption}
\newtheorem*{claim}{Claim}
\newtheorem{answer}[equation]{Answer}
%%%%%% letters %%%%

\newcommand{\fa}{\mathfrak{a}}
\newcommand{\fb}{\mathfrak{b}}
\newcommand{\fm}{\mathfrak{m}}
\newcommand{\fp}{\mathfrak{p}}
\newcommand{\fq}{\mathfrak{q}}

\newcommand{\IA}{\mathbb{A}}
\newcommand{\IN}{\mathbb{N}}
\newcommand{\IF}{\mathbb{F}}
\newcommand{\IP}{\mathbb{P}}
\newcommand{\IZ}{\mathbb{Z}}

\newcommand{\sD}{\mathcal{D}}
\newcommand{\sI}{\mathcal{I}}
\newcommand{\sO}{\mathcal{O}}
\newcommand{\sP}{\mathcal{P}}
\newcommand{\sQ}{\mathcal{Q}}
\newcommand{\sT}{\mathcal{T}}
\newcommand{\sU}{\mathcal{U}}

\newcommand{\shF}{\mathscr{F}}
\newcommand{\shG}{\mathscr{G}}
%%%%%%% macros %%%%%

%% my definitions %%%

\newcommand{\End}{\mathrm{End}}
\newcommand{\tr}{\mathrm{tr}}
\newcommand{\Hom}{\mathrm{Hom}}
\newcommand{\Aut}{\mathrm{Aut}}
\newcommand{\Trace}{\mathrm{Trace}\,}
\newcommand{\rank}{\mathrm{rank}}
\renewcommand{\deg}{\mathrm{deg}\,}
\newcommand{\Spec}{\rm Spec\,}
\newcommand{\Proj}{\rm Proj\,}
\newcommand{\Sym}{\mathrm{Sym \,}}
\newcommand{\Span}{\mathrm{Span \,}}
\renewcommand\dim{{\rm dim\,}}
\newcommand{\codim}{{\rm codim\,}}
\renewcommand\det{{\rm det\,}}
\newcommand{\im}{{\rm Im\,}}


\newcommand\iso{{\, \simeq \,}} 
\newcommand\tensor{{\otimes}}
\newcommand\Tensor{{\bigotimes}} 
\newcommand\union{\bigcup} 
\newcommand\onehalf{\frac{1}{2}}
\newcommand\trivial{{\mathbb I}}
\newcommand\wb{\overline}

%%%%%Delimiters%%%%

\newcommand{\<}{\langle}
\renewcommand{\>}{\rangle}

%\renewcommand{\(}{\left(}
%\renewcommand{\)}{\right)}


%%%% Different kind of derivatives %%%%%

\newcommand{\delbar}{\bar{\partial}}
\newcommand{\pdu}{\frac{\partial}{\partial u}}
%\newcommand{\pd}[1][2]{\frac{\partial #1}{\partial #2}}

%%%%% Arrows %%%%%
\newcommand{\induce}{\rightsquigarrow}
\newcommand{\into}{\hookrightarrow}
\newcommand{\onto}{\twoheadrightarrow}
\newcommand{\tto}{\longmapsto}
\def\llra{\longleftrightarrow}
\def\wt{\widetilde}
\def\wtilde{\widetilde}
\def\what{\widehat}
\def\bf{\textbf}
\def\it{\textit}
%%%%%%%%%%%%%%%%%%% Ziquan's definitions %%%%%%%%%%%%%%%%%%%%
\newcommand{\Ann}{\mathrm{Ann}}
\newcommand{\height}{\mathrm{height \,}}
\newcommand{\Div}{\mathrm{Div}}
\newcommand{\sE}{\mathcal{E}}
\newcommand{\p}{\partial}
\newcommand{\Ohm}{\Omega}
\newcommand{\w}{\omega}
\newcommand{\sing}{\mathrm{sing}}
%%%%%%%%%%%%% new definitions for the positive mass paper %%%%%%%%%

\newcommand{\sperp}{{\scriptscriptstyle \perp}}
\newcommand{\Qmed}{\mathcal{Q}_r^\mathrm{medium}}
\newcommand{\Qhigh}{\mathcal{Q}^\mathrm{high}}
\newcommand{\uppermu}{\overline{\mu}}
\newcommand{\lowermu}{\underline{\mu}}
\newcommand{\res}{\mathrm{res}}
\newcommand{\ev}{\mathrm{ev}}
\newcommand{\pr}{\mathrm{pr}}
\newcommand{\Prob}{\mathrm{Prob}}
\newcommand{\st}{\, \mathrm{ s.t. }\,}

\newcommand{\bq}{\textbf{q}}
%%%%%%%%%%%%%%%%%%%%%%%

%%%%%%%%%%%%%%%%%%%%%%%%%%%%%%%%%%%%%%%%%%%%%



%
\begin{document}
%

\title{Simply Ramified Curves in $\IP^1 \times \IP^1$}
\author{Ziquan Yang}


\date{}

\maketitle

 
%\setcounter{secnumdepth}{1} 

\setcounter{section}{0}
\section{Introduction}
How should I do the introduction?
\paragraph{Notation} 
Let $X = \IP^1 \times \IP^1$ over a ground field $\kappa$. We label its coordinates as $((s_0, s_1), (t_0, t_1))$. When we need to differentiate, the first copy of $\IP^1$ is labelled $\IP^1_s$ and the second one $\IP^1_t$. Let $\pi : X \to \IP^1_t$ be the natural projection map. Given $P \in \IP^1_t$, let $X_Q$ denotes the fiber $\pi^{-1} P$. $R_{n, d} = H^0(\IP^1 \times \IP^1, \sO(n, d))$. For a section $f \in R_{n, d}$ or $\IP R_{n, d}$, the hypersurface in $X$ cut out by $f$ is denoted by $H_f$. If $C \subseteq X$ is a curve and $Q \in X$ is a closed point, then $e_Q(C)$ denotes the ramification degree of $C$ with respect to $\pi$. 
\section{An analogue of Bertini's theorem} 
\begin{theorem}
When $\kappa$ is algebraically closed, then for almost all $f \in R_{n, d}$, $H_f$ is simply ramified. More precisely, let $P^0$ be the subscheme of $\IP R_{n, d}$ parametrizing non-singular curves, then the subset 
$$ D = \{ f \in \IP R_{n, d} : H_f \textit{ is not simply ramified w.r.t. } \pi \}$$ is contained in a proper closed subscheme of $P^0$. 
\end{theorem} 
\begin{proof}
To simplify notation in this proof we write $V = R_{n, d}$. 
There are two types of curves in $D$. Type I curves are those that have a ramfication degree $\ge 3$ at some point. Type II curves are those that ramify at more than $2$ points along some fiber. We want to define subschemes of $\IP V$ that can be informally described as  
\begin{align*} 
D_1 &= \{ f \in \IP V : \exists Q \in X, \st e_Q(H_f) \ge 3 \} \\
D_2 &= \{ f \in \IP V : \exists P \in \IP^1_t, \st e_{Q}(H_f), e_{Q'}(H_f) \ge 2 \textit{ for some }Q \neq Q' \in X_P \}
\end{align*}
Clearly $D$ corresponds the set of closed points in the scheme $P^0 \cap (D_1 \cup D_2)$ and hence it suffices to prove the following two claims. \\\\
\textit{Claim 1 :} $D_1$ is a proper closed subscheme of $\IP V$. \\
This is more or less a standard application of Bertini's arguments. We want to define a subscheme $\wt{D}_1 \subseteq X \times \IP V$ to parametrize pairs $(P, f)$ such that $e_Q(H_f) \ge 3$. Suppose $Q \in \IA^1 \times \IA^2 = \Spec \kappa[s, t] \subset X$ where $s = s_0/s_1, t=t_0/t_1$. Then $f \in \IP V$ can be written in the form
$$ f = \sum_{0 \le j \le n} \sum_{0 \le i \le d} a_{ij} s^i t^j $$
where $(a_{ij})$ is homogeneous coordinates for $\IP V$. $e_Q(H_f) \ge 3$ if and only if
$$ f(Q) = \frac{ \p }{\p s } f (Q) = \frac{ \p^2 }{\p s^2 } f(Q) = 0 $$
Hence on $\IA^1 \times \IA^1 \times X$, $\wt{D}_1$ can be described as $\{ f = f_s = f_{ss} = 0 \}$ . Clearly the $e_Q(H_f)$ is independent of the affine chart of $X$ that we choose to cover $Q$, so if we define $\wt{D}_1$ this way on other charts, they patch up. For each $Q \in X$, $f \in (\wt{D}_1)_Q$ amounts to $3$ linear conditions on $\IP V$. Therefore we can easily compute that $\dim \wt{D}_1 \le \IP V - 1$. Let $D_1$ be the image of $\wt{D}_1$ under the projection to $\IP V$. $\wt{D}_1$ is closed in $X \times \IP V$ and hence by elimination theory $D_1$ is closed in $\IP V$. It is proper since $\dim D_1 \le \IP V - 1$.  \\\\
\textit{Claim 2 :} $D_2$ is a contained in a proper closed subscheme of $\IP V$. \\
The proof is not as straightforward as the previous one, since the question is no longer local at a point of $X$. Nonetheless we may sieve on pairs of points along the same fiber. Consider the space $ X \times X$. We label its coordinates as $((x_0 : x_1), (y_0 : y_1), (s_0 : s_1), (t_0 : t_1))$. Let $T'$ be the subscheme defined by $y_0 t_1 - y_1 t_0 = 0$ and let $\Delta$ be the diagonal of $ X \times X$, i.e. $\{y_0 t_1 - y_1 t_0 = 0, x_0 s_1 - x_1 s_0 = 0 \}$. Define $T = T' - \Delta$. Clearly $T$ is simply parametrizing pairs on $X$ that project to the same point under $\pi$. This time we let $B$ be the subscheme of $ T \times \IP V$ parametrizing pairs $((Q, Q'), f)$ such that $e_Q(H_f),  e_{Q'}(H_f) \ge 2$. More precisely, on $(\IA^1 \times \IA^1) \times (\IA^1 \times \IA^1) \times \IP V$, where 
$(\IA^1 \times \IA^1) \times (\IA^1 \times \IA^1) \subset X \times X$ is the affine chart with coordinates $x= x_0/x_1, y = y_0/y_1, s = s_0/s_1, t = t_0/t_1$, $\wt{D}_2$ is defined as 
\begin{equation}\tag{*} f(x , y) = \frac{\p}{\p x}f(x, y) = f(s, t) = \frac{\p}{\p s}f(s, t) = 0 \end{equation}
To make sense of $f(x, y)$, we simply associate the point $f \in \IP V$ with homogeneous coordinates $(a_{ij})$ with $$ f = \sum_{0 \le j \le n} \sum_{0 \le i \le d} a_{ij} x^i y^j $$
Therefore (*) is nothing but a concise way to write down polynomial equations in $x, y, s, t$ and $a_{ij}$'s. For each point $(Q, Q') \in X \times X$, $f \in (\wt{D}_2)_{(Q, Q')}$ amounts to 4 linear conditions on $a_{ij}$'s. Therefore we easily obtain $\dim \wt{D}_2 = \IP V - 1$. $D_2$ is defined as the image of $\wt{D}_2$ under the projection to $\IP V$, and there $D_2$ is a locally closed subscheme of $\IP V$ with closure $\overline{D}_2 \neq \IP V$. This proves Claim 2. 


\end{proof}

\section{A density result}
If $\sP \subseteq \union_d R_{3, d}$ is subset, we define 
$$ \Prob(f \in \sP) = \lim_{d \to \infty}\Prob(f_d \in \sP) $$
where $f$ and $f_d$ are randomly chosen from $\union_{d} R_{3, d}$ and $R_{3, d}$ respectively. 
\begin{theorem}
\label{main}
Suppose $p = \mathrm{char } \IF_q > 2$. Let $\sD \subset \union_d R_{3,d}$ be the subset of sections $f$ such that $H_f$ is simply ramified with respect to $\pi$. Then
$$ \Prob(f \in \sD) = \zeta_{\IP_{\IF_q}}(2)^{-2} $$ In particular, the conditional probability for a randomly chosen non-singular curve to be simply ramified is $\zeta_{\IP_{\IF_q}}(2)/\zeta_{\IP_{\IF_q}}(3)$. 
\end{theorem}

We say $f \in R_{n, d}$ is ``good" at $Q \in X$, if $H_f$ is both non-singular and simply ramified at $Q$, and ``bad" otherwise. For a fixed $e_0 \in \IN$, we define 
\begin{align*}
\sP_{e_0}^{\mathrm{low}} &= \union_{d \ge 0} \{ f \in R_{3, d} : f \textit{ is good at all $Q \in X$,}\deg \pi(Q) < e_0\}\\
\sQ_{e_0}^{\mathrm{med}} &= \union_{d \ge 0} \{f \in R_{3, d} : f \textit{ is bad at some $Q$, }\deg \pi(Q) \in [e_0, \lfloor d/p \rfloor]\}\\
\sQ^{\mathrm{high}} &= \union_{d \ge 0} \{f \in R_{3, d} : f \textit{ is bad at some $Q$, }\deg \pi(Q) > d/p\}
\end{align*}

\begin{lemma}
$$\Prob(f \in \sP_{e_0}^{\mathrm{low}}) = \prod_{\deg(P) < e_0} \Prob(H_f \textit{ is good at all points of } \pi^{-1}(P))$$
\end{lemma}

Now we want to do some local analysis on the fiber. Let $P \in \IP^1_t$ be a closed point. Without loss of generality, we assume $t_1 \neq 0$ at $P$ and and lies in $\IA^1 = \Spec \IF_q[t]$ where $t = t_0/t_1$.   Let $\fm = (r(t))\subseteq \IF_q[t]$ be the maximal ideal corresponding to $P$, so that $P^{(2)} = \Spec \IF_q[t]/\fm^2$ and $\kappa(P) = \IF_q[t]/\fm$. Denote the reduction map $\IF_q[t]/\fm^2 \to \IF_q[t]/\fm$ by $g \mapsto \overline{g}$. We extend it to a map $H^0(\IP^1_{\IF_q[t]/\fm^2}, \sO(3)) \to H^0(\IP^1_{\IF_q[t]/\fm}, \sO(3))$ by applying $\IF_q[t]/\fm^2 \to \IF_q[t]/\fm$ to each coefficients. Let $\varphi_P : R_{3, d} \to H^0(\IP^1_{\IF_q[t]/\fm^2}, \sO(3))$ be the restriction map. $H_f$ is smooth at a point $Q \in \pi^{-1}(P)$ if and only if $\varphi_P(f)$ does not vanish at $Q^{(2)} \in \IP^1_{\IF_q[t]/\fm^2}$, and $e_Q(H_f) \ge 3$ if and only if $\wb{\varphi_P(f)}$ does not have a zero of multiplicity $\ge 3$ at $Q$ on $\IP^1_{\IF_q[t]/\fm}$. In other words, we can determine if $H_f$ is good at all points on $\pi^{-1}(P)$ by looking at the image of $f$ in $H^0(\IP^1_{\IF_q[t]/\fm^2}, \sO(3))$, so we observe that:

\begin{lemma}
\label{Low}
$$\Prob(f \in \sP_{e_0}^{\mathrm{low}})  = \prod_{\deg(P) < e_0}  \Prob( H_f \textit{ is good at all points } Q \in H_f \cap X_P ) $$
\end{lemma}
\begin{proof}
More generally, if $\{P_1, P_2, \cdots, P_s\}$ is a set of finitely many closed points in $\IP^1$, the restriction map 
$$ R_{3, d} = H^0(\IP^1 \times \IP^1, \sO(3, d)) \to \prod_{i = 1}^s H^0(X_{P_i^{(2)}}, \sO(3))$$
is surjective for $ d \ge \sum_i 2 \deg P_i + 1$.  
\end{proof}

We make the following convention:
 We call a pair $(F_1, F_2) \in H^0(\IP^1_{\kappa(P)}, \sO(3))^2$ ``bad" if it is one of the following 3 types:
\begin{enumerate}
\item $F_1$ has a root of multiplicity $\ge 2$ at a point where $F_2$ also vanishes. 
\item $F_1$ has a root of multiplicity $3$ at a point where $F_2$ does not vanish. 
\item $F_1 \equiv 0$.
\end{enumerate}
Let $\, \wt{} : \IF_q[t]/\fm \to \IF_q[t]/\fm^2$ be a $\IF_q$-linear map that is a section to the reduction map $\, \bar{} : \IF_q[t]/\fm^2 \to \IF_q[t]/\fm$. We extend it to a map $H^0(\IP^1_{\IF_q[t]/\fm^2}, \sO(3)) \to H^0(\IP^1_{\IF_q[t]/\fm}, \sO(3))$ in the same way we extended the reduction map. Then we have the following:
\begin{lemma}
$$ H^0(\IP^1_{\IF_q[t]/\fm}, \sO(3))^2 \to H^0(\IP^1_{\IF_q[t]/\fm^2}, \sO(3)) $$
given by $(F_1, F_2) \mapsto \wt{F}_1 + uF_2$ is a bijection. Moreover, $f \in R_{3, d}$ is ``good" at all points $Q \in \pi^{-1}(P)$ if and only if $\varphi_P(f)$ does not correspond to a bad pair. 
\end{lemma}
\begin{proof}
The map $f \mapsto ( \wb{f}, (f - \wt{\wb{f}})/r(t))$ is an inverse to the map given in the lemma. Let $(F_1, F_2)$ be the pair corresponding to $\varphi_P(f)$. $\varphi_P(f)$ vanishes on $Q^{(2)}$, i.e. $H_f$ is singular at $Q$, if and only if $Q$ is a double root of $F_1$ and a root to $F_2$. Similarly, $e_Q(H_f) \ge 3$ if and only if $Q$ is root to $F_1$ of multiplicity $\ge 3$. 
\end{proof}

\begin{lemma}
\label{count}
The density of good pairs in $H^0(\IP^1_{\kappa(P)}, \sO(3))^2$ is 
$$ (1 - q^{-2e})^{2} $$
\end{lemma}
\begin{proof}
We first count type 1 pairs. Let $P$ be the point that is a double root to $F_1$ and a root to $F_2$. Note that $\deg P = 1$. $F_1$ is fixed up to rescaling by $\IF_{q^e}^*$ after we choose a third root, which can be any point of degree $1$. Therefore there are $(q^e + 1)(q^e - 1)$ choices for $F_1$. The probability that $F_2$ vanishes at $P$ is $q^{-e}$, so we have $q^3$ choices for $F_2$. Since we have $(q^e + 1)$ choices for $P$, there are $q^{3e}(q^e + 1)^2(q^e - 1)$ type 1 pairs. 

To count type 2 pairs, let $P$ be the triple root to $F_1$. Again there are $q^e + 1$ choices. At each $P$, there are $(q^e -1)$ homogeneous polynomials of degree $3$ which have $P$ as a triple root, since the leading coefficient will uniquely determine such a polynomial. We have $q^{4e} - q^{3e}$ choices for $F_2$. In total there are $ (q^e + 1)(q^e - 1)(q^{4e} - q^{3e})$
pairs of type 2. 

Finally when $F_1 = 0$, $F_2$ can be anything, so we have $q^{4e}$ type 3 pairs.
Therefore the density of good pairs is 
$$ 1 - q^{-8e}(q^{3e}(q^e + 1)^2(q^e - 1) + (q^e + 1)(q^e - 1)(q^{4e} - q^{3e}) + q^{4e}) = (1 - q^{-2e})^2 $$ 
\end{proof}


\begin{lemma}
\label{Medium}
$$\lim_{e_0 \to \infty} \Prob( f \in  \sQ_{e_0}^{\mathrm{med}}) = 0 $$
\end{lemma}
\begin{proof}
Let $P$ be a point of degree $e < \lfloor d/p \rfloor$ on $C$ and let $B_P$ denote the event that a randomly chosen $f \in R_{3, d}$ is ``bad" at some point in the fiber $X_P$. By the proof of Lemma~\ref{Low}, the restriction map $R_{3, d} \to H^0(P^{(2)}, \sO(3))$ is surjective since $p > 2$ and $e < \lfloor d/p \rfloor < (d-1)/2$ when $d$ is large. Lemma~\ref{count} hence implies that probability that $f$ is ``bad" at some point in the fiber $X_P$ is $1 - (1 - q^{-2e})^2 < 2q^{-2e}$. 
\begin{align*}
\Prob(f \in \sQ_{e_0}^{\mathrm{med}}) &\le \sum_{e = e_0}^{\lfloor d/p \rfloor} (\text{number of points of degree $e$ in $\IP^1$})(2q^{-2e}) \\
&\le O(\sum_{e = e_0}^{\lfloor d/p \rfloor}q^{e} q^{-2e})\\
&= O(\frac{c q^{-e_0}}{1 - q^{-1}})
\end{align*}
Therefore as $e_0 \to \infty$, $\Prob(f \in \sQ_{e_0}^{\mathrm{med}}) \to 0$. 
\end{proof}

\begin{lemma}
\label{prep}
Let $j > 2$ be an integer. For a randomly chosen $f \in R_{3, d}$, the probability that there exists a point $Q \in \IP^1 \times \IP^1$ with $\deg \pi(Q) \ge j$ such that $e_{Q}(H_f) \ge 3$ or $H_f$ is singular at $Q$ is at most 
$$ O(d^2 q^{- \min(\lfloor d/p \rfloor + 1, j)}) $$
\end{lemma}
\begin{proof}
Since $\IP^1 \times \IP^1$ can be covered by $4$ affine charts $\IA^1 \times \IA^1$, it suffices for us to show that for each $\IA^1 \times \IA^1$, we have that for a randomly chosen $f \in R_{3, d}$, the probability that there exists a point $Q \in \IA^1 \times \IA^1$ with $\deg \pi(Q) \ge j$ such that $e_{Q}(H_f) \ge 3$ or $H_f$ is singular at $Q$ is at most $O(d^2 q^{- \min(\lfloor d/p \rfloor + 1, j)})$. Therefore we may work affine-locally. 

Without loss of generality, we may assume $s_1 \neq 0, t_1 \neq 0$ on $\IA^1 \times \IA^1$ and work with coordinates $s = s_0/s_1, t = t_0/t_1$. Let $A_{n, d}$ be the polynomials that are of degree $\le n$ in $s$ and $\le d$ in $t$. Clearly by dehomogenizing sections in $S_{3, d}$, we may naturally identify $S_{3, d} = A_{3, d}$. Accordingly, we replace $H_f$ by $H_f \cap \IA^1 \times \IA^1$. We call a closed point $Q \in \IA^1 \times \IA^1$ \textit{admissible} if $\deg \pi(Q) \ge j$ and a subscheme $W \subseteq \IA^1 \times \IA^1$ \textit{admissible} if it contains an admissible point. By $(W)_{\mathrm{ad}}$ we denote the union of admissible irreducible component of $(W)_{\mathrm{red}}$. 

We first deal with the probability that for a randomly chosen $f \in A_{3, d}$, $e_Q(H_f) \ge 3$ for some admissible $Q \in \IA^1 \times \IA^1$, which happens if and only if \begin{equation} \label{ramify} f(Q) = f_s(Q) = f_{ss}(Q) = 0 \end{equation}  

Define $W_2 = \{f_{ss} = 0 \}, W_1 = W_2 \cap \{ f_s = 0 \}$ and $W_0 = W_1 \cap \{ f = 0 \}$. We want to bound the probability that for a randomly chosen $f \in A_{3, d}$, $W_0$ contains an admissible point.  

Following Poonen's idea, we write $f$ is such a way so that the first and second order partial derivatives are largely independent. If $f_0 \in A_{3, d}$ and $g_1, g_2, h \in A_{0, \lfloor d/p \rfloor}$ are selected uniformly and independently at random, then the distribution of 
$$ f = f_0 + g_1^p s^2 + g_2^p s + h^p $$
is uniform over $A_{3, d}$. 
Direct computation shows that 
\begin{align*}
f_s &= f_{0, s} + 2 g_1^p s + g_2^p\\
f_{ss} &= f_{0, ss} + 2 g_1^p 
\end{align*} 
Note that $W_2$ depends only on the choice of $f_0, g_1$ and $W_1$ only on $f_0, g_1, g_2$. Let $E$ denote the event that 
\begin{enumerate}[a.]
\item The \textit{admissible} irreducible components of $W_1$ are of dimension $0$. 
\item $f$ does not vanish at any of these irreducible components. 
\end{enumerate} 
Clearly if $E$ holds for $f$, then $W_0$ does not contain any admissible point. Therefore it suffices to show that for a randomly chosen $f \in A_{3, d}$, 
$$ \Prob(E) = 1 - O(d^2 q^{- \min(\lfloor d/p \rfloor + 1, j)})  $$
as $d \to \infty$. Now we bound $\Prob(E)$ in three steps: \\\\
\textit{Step 1: }Conditioned on a choice of $f_0$, the probability that $\dim W_2 = 2$ is at most $q^{- (\lfloor d/p \rfloor + 1)}$, since $\dim W_2 = 2$ if and only if $g_1^p = - f_{0, ss}/2$, for which there is at most one choice of $g_1$. \\\\
\textit{Step 2: }Conditioned on a choice of $f_0$ and $g_1$ such that $\dim W_2 = 1$, the probability that $\dim (W_1)_{\mathrm{ad}} = 1$ is at most $O(d q^{- \min(\lfloor d/p \rfloor + 1, j)})$. Let $V_1, \cdots, V_\ell$ be the $\IF_q$-irreducible component of $\{W_2 \}_{\mathrm{red}}$. View $\IP^1 \times \IP^1$ as a subscheme of $\IP^3$ via Segre embedding, we may apply B{\'e}zout's theorem to obtain that $\ell = O(d)$. $\dim W_1 = 1$ and $\pi(W_1)$ contains a point $P$ with $\deg P \ge j$ if and only if $f_s$ vanishes identically on $V_i$ for some $i$. We need to bound the set
$$ G^{\mathrm{bad}}_i = \{ g_2 \in A_{0, \lfloor d/p \rfloor} : f_{0, s} + 2 g_1^p s + g_2^p \textit{ vanishes identically on }V_i \} $$
If $g, g' \in G^{\mathrm{bad}}_i$, then $g - g'$ vanihes identically on $V_i$. Therefore $G^{\mathrm{bad}}_i$ is a coset of $\ker \varphi_i$, where $\varphi_i$ is the $\IF_q$-linear map $\varphi_i : A_{0, \lfloor d/p \rfloor} \to H^0(V_i, \sO_{V_i})$. Now we divide it into two cases. 

Case 1: If $\dim \pi(V_i) > 0$, then the function $t$, and hence any nonzero polynomial in $t$, does not vanish identically on $V_i$. Therefore the codimension of $\ker \varphi_i$ in $A_{0, \lfloor d/p \rfloor}$ is $\lfloor d/p \rfloor + 1$, and the probability that $f_s$ vanishes identically on $V_i$ is at most $q^{- (\lfloor d/p \rfloor + 1)}$.   

Case 2: If $\dim \pi(V_i) = 0$, then since $V_i$ is assumed to be admissible,  $\deg \pi(V_i) \ge j$ . $\varphi_i$ factors as $$A_{0, \lfloor d/p \rfloor} = \IF[t]_{\le \lfloor d/p \rfloor} \to H^0(\pi(V_i), \sO_{\pi(V_i)}) \stackrel{\pi^*}{\to} H^0(V_i, \sO_{V_i})$$ The pullback map $\pi^*$ is clearly injective. Let $B_{0, i}$ be the image of $A_{0, i}$ in $H^0(\pi(V_i), \sO_{\pi(V_i)})$. Suppose $\pi(V_i) = \Spec \IF_q[t]/(r(t))$, where $\deg r(t) = \deg \pi(V_i)$. Then it is clear that $\dim_{\IF_q} B_{0, i}$ increases in dimension with each increase of $i$ until it stabilizes at $\deg r$. Therefore $\dim \im \varphi_i \ge \min(\lfloor d/p \rfloor + 1, j)$, and the probability that $f_s$ vanishes identically on $V_i$ is at most $q^{- \min(\lfloor d/p \rfloor + 1, j)}$.   
Since $q^{- (\lfloor d/p \rfloor + 1)} \le q^{- \min(\lfloor d/p \rfloor + 1, j)}$, and there are at most $O(d)$ such components $V_i$, in either case we obtain the desired conclusion of this step. \\\\
\textit{Step 3: }Conditioned on a choice of $f_0, g_1$ and $g_2$ such that $\dim (W_1)_{\mathrm{ad}} = 0$, the probability that $(W_0)_{\mathrm{ad}} \neq \emptyset$ is at most $O(d^2 q^{- \min(\lfloor d/p \rfloor + 1, j)})$. Let $Q_1, Q_2, \cdots, Q_r$ be all irreducible components of $(W_1)_{\mathrm{ad}}$, where $r = |(W_1)_{\mathrm{ad}}|$. Since $W_1$ is cut out by $f_s$ and $f_ss$, and $\deg f_s, \deg f_{ss} = O(d)$, by B{\'e}zout theorem $r = O(d^2)$, and the same argument as in the previous paragraph shows that at each point in $W_1$, the probability that $f = 0$ at the point is at most $q^{- \min(\lfloor d/p \rfloor + 1, j)}$. \\\\
Finally \textit{Step 1} and \textit{2} combined to give that 
$$ \Prob(E_a) \ge (1 - q^{- (\lfloor d/p \rfloor + 1)})(1- q^{- \min(\lfloor d/p \rfloor + 1, j)}) = 1 - O(d q^{- \min(\lfloor d/p \rfloor + 1, j)}) $$
And \textit{Step 3} gives 
$$ \Prob(E) \ge \Prob(E_a)(1 - O(d^2 q^{- \min(\lfloor d/p \rfloor + 1, j)})) = 1 - O(d^2 q^{- \min(\lfloor d/p \rfloor + 1, j)}) $$

Now we deal with the probability that $H_f$ is singular at $Q$ for some $Q \in \IA^1 \times \IA^1$, which happens if and only if $f(Q) = f_s(Q) = f_t(Q) = 0$. This time we may write $f$ in the form 
$$ f = f_0 + g_1^p s + g_2^p t + h^p $$
for some randomly chosen $f_0 \in A_{3, d}$ and $g_1, g_2, h \in A_{0, \lfloor d/p \rfloor}$. The rest of the proof is completely analogous to the above. 
\end{proof}


\begin{lemma}
\label{High}
$$\Prob( f \in  \sQ^{\mathrm{high}}) = 0 $$
\end{lemma}
\begin{proof}
Apply Lemma~\ref{prep} with $j = \lfloor d/p \rfloor$. 
\end{proof}
\noindent \textit{Proof of Theorem~\ref{main}.} 
For each $e_0$, we have that 
$$  \sP_{e_0}^{\mathrm{low}} \subseteq \sD \subseteq \sP_{e_0}^{\mathrm{low}} \cup \sQ_{e_0}^{\mathrm{med}} \cup \sQ^{\mathrm{high}}$$
Therefore 
$$ \Prob(f \in \sP_{e_0}^{\mathrm{low}}) \le \Prob(f \in \sD) \le \Prob(f \in \sP_{e_0}^{\mathrm{low}} \cup \sQ_{e_0}^{\mathrm{med}} \cup \sQ^{\mathrm{high}})$$
Now take $e_0 \to \infty$, Lemma~\ref{Low}, \ref{count}, \ref{Medium}, \ref{High} combine the give the result. \qed
\begin{thebibliography}{9}
\bibitem{Wood} 
D. Erman and M.M. Wood, \textit{Semiample Bertini theorems over finite fields}, Duke Mathematical Journal 164 (2015), no. 1, 1-38

\bibitem{Poonen}
B. Poonen, \textit{Bertini theorems over finite fields}, Ann. of Math. (2) 160 (2004), no. 3, 1099-1127.
\end{thebibliography}

\end{document}
