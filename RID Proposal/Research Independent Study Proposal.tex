\documentclass[12pt]{article}
\usepackage{amsmath}
\usepackage{amsthm}
\usepackage{amsfonts}
\usepackage{amssymb}
\usepackage{latexsym}
\usepackage{tikz} 
\usepackage{esint} 






\usepackage[matrix,tips,graph,curve]{xy}



\newcommand{\mnote}[1]{${}^*$\marginpar{\footnotesize ${}^*$#1}}
\linespread{1.065}

\makeatletter

\setlength\@tempdima  {5.5in}
\addtolength\@tempdima {-\textwidth}
\addtolength\hoffset{-0.5\@tempdima}
\setlength{\textwidth}{5.5in}
\setlength{\textheight}{8.75in}
\addtolength\voffset{-0.625in}

\makeatother

\makeatletter 
\@addtoreset{equation}{section}
\makeatother


\renewcommand{\theequation}{\thesection.\arabic{equation}}

\theoremstyle{plain}
\newtheorem{theorem}[equation]{Theorem}
\newtheorem{corollary}[equation]{Corollary}
\newtheorem{conjecture}[equation]{Conjecture}
\newtheorem{lemma}[equation]{Lemma}
\newtheorem{proposition}[equation]{Proposition}
\theoremstyle{definition}
\newtheorem{definition}[equation]{Definition}
\newtheorem{definitions}[equation]{Definitions}
%\theoremstyle{remark}

\newtheorem{remark}[equation]{Remark}
\newtheorem{remarks}[equation]{Remarks}
\newtheorem{exercise}[equation]{Exercise}
\newtheorem{example}[equation]{Example}
\newtheorem{examples}[equation]{Examples}
\newtheorem{notation}[equation]{Notation}
\newtheorem{question}[equation]{Question}
\newtheorem{assumption}[equation]{Assumption}
\newtheorem*{claim}{Claim}
\newtheorem{answer}[equation]{Answer}
%%%%%% letters %%%%

\newcommand{\fa}{\mathfrak{a}}
\newcommand{\fb}{\mathfrak{b}}
\newcommand{\fm}{\mathfrak{m}}
\newcommand{\fp}{\mathfrak{p}}
\newcommand{\fq}{\mathfrak{q}}

\newcommand{\IA}{\mathbb{A}}
\newcommand{\IN}{\mathbb{N}}
\newcommand{\IF}{\mathbb{F}}
\newcommand{\IP}{\mathbb{P}}
\newcommand{\IZ}{\mathbb{Z}}

\newcommand{\sD}{\mathcal{D}}
\newcommand{\sI}{\mathcal{I}}
\newcommand{\sO}{\mathcal{O}}
\newcommand{\sP}{\mathcal{P}}
\newcommand{\sT}{\mathcal{T}}
\newcommand{\sU}{\mathcal{U}}

\newcommand{\shF}{\mathscr{F}}
\newcommand{\shG}{\mathscr{G}}
%%%%%%% macros %%%%%

%% my definitions %%%

\newcommand{\End}{\mathrm{End}}
\newcommand{\tr}{\mathrm{tr}}
\newcommand{\Hom}{\mathrm{Hom}}
\newcommand{\Aut}{\mathrm{Aut}}
\newcommand{\Trace}{\mathrm{Trace}\,}
\newcommand{\rank}{\mathrm{rank}}
\renewcommand{\deg}{\mathrm{deg}\,}
\newcommand{\Spec}{\rm Spec\,}
\newcommand{\Proj}{\rm Proj\,}
\newcommand{\Sym}{\mathrm{Sym \,}}
\newcommand{\Span}{\mathrm{Span \,}}
\renewcommand\dim{{\rm dim\,}}
\newcommand{\codim}{{\rm codim\,}}
\renewcommand\det{{\rm det\,}}
\newcommand{\im}{{\rm Im\,}}


\newcommand\iso{{\, \simeq \,}} 
\newcommand\tensor{{\otimes}}
\newcommand\Tensor{{\bigotimes}} 
\newcommand\union{\bigcup} 
\newcommand\onehalf{\frac{1}{2}}
\newcommand\trivial{{\mathbb I}}
\newcommand\wb{\overline}

%%%%%Delimiters%%%%

\newcommand{\<}{\langle}
\renewcommand{\>}{\rangle}

%\renewcommand{\(}{\left(}
%\renewcommand{\)}{\right)}


%%%% Different kind of derivatives %%%%%

\newcommand{\delbar}{\bar{\partial}}
\newcommand{\pdu}{\frac{\partial}{\partial u}}
%\newcommand{\pd}[1][2]{\frac{\partial #1}{\partial #2}}

%%%%% Arrows %%%%%
\newcommand{\induce}{\rightsquigarrow}
\newcommand{\into}{\hookrightarrow}
\newcommand{\onto}{\twoheadrightarrow}
\newcommand{\tto}{\longmapsto}
\def\llra{\longleftrightarrow}
\def\wt{\widetilde}
\def\wtilde{\widetilde}
\def\what{\widehat}
\def\bf{\textbf}
\def\it{\textit}
%%%%%%%%%%%%%%%%%%% Ziquan's definitions %%%%%%%%%%%%%%%%%%%%
\newcommand{\Ann}{\mathrm{Ann}}
\newcommand{\height}{\mathrm{height \,}}
\newcommand{\Div}{\mathrm{Div}}
\newcommand{\sE}{\mathcal{E}}
\newcommand{\p}{\partial}
\newcommand{\Ohm}{\Omega}
\newcommand{\sing}{\mathrm{sing}}
%%%%%%%%%%%%% new definitions for the positive mass paper %%%%%%%%%

\newcommand{\sperp}{{\scriptscriptstyle \perp}}
\newcommand{\Qmed}{\mathcal{Q}_r^\mathrm{medium}}
\newcommand{\Qhigh}{\mathcal{Q}^\mathrm{high}}
\newcommand{\uppermu}{\overline{\mu}}
\newcommand{\lowermu}{\underline{\mu}}
\newcommand{\res}{\mathrm{res}}
\newcommand{\ev}{\mathrm{ev}}
%%%%%%%%%%%%%%%%%%%%%%%

%%%%%%%%%%%%%%%%%%%%%%%%%%%%%%%%%%%%%%%%%%%%%



%
\begin{document}
%

\title{Research Independent Study Proposal}
\author{Ziquan Yang}


\date{August 24th, 2015}

\maketitle


 
%\setcounter{secnumdepth}{1} 

\setcounter{section}{0}
\section{Title and Description of Proposed Study}
\paragraph{Title} Sieve methods in algebraic geometry
\paragraph{Background}
The problems that I am working on are variants of Bertini theorems of schemes over finite fields. Roughly speaking, classical Berini type theorems say that over an infinite field $k$ if a subscheme $X \subseteq \IP^n$ has a certain property (smooth, irreducible, etc), then a sufficiently general hyperplane section over $k$ has the same property. In extending the Bertini smoothness theorem to schemes over finite fields, Poonen \cite{PBertini} replaced hyperplanes with hypersurfaces and computed the asymptotic probability that a randomly chosen hypersurface of degree $d$ has a smooth intersection with $X$ as $d \to \infty$. Let $f \in H^0(\IP^n, \sO(d))$ be randomly chosen, and let $H_f$ denote the corresponding hypersurface section on $X$. What Poonen actually showed is essentially an independence result:
$$ \lim_{d \to \infty} \mathrm{Prob}(H_f \textit{ is smooth}) = \prod_{P \in X} \mathrm{Prob}(H_f \textit{ is smooth at  }P)$$ 

We could have rephrased the question to appear more intrinsic of $X$ using the language of divisors. Let $A$ be the very ample divisor that induces the embedding $X \subseteq \IP^n$, hypersurface sections are parametrized by $H^0(X, \sO(dA))$. Building on Poonen's work, Erman and Wood extended the result further to semiample divisors in \cite{Wood}, more specifically those of the from $nA + dE$, where $A$ is very ample and $E$ is only guaranteed to be globally generated. If $\pi : X \to \IP^M$ is the morphism to projective space induced by $E$ and $n$ is large enough, then 
$$ \lim_{d \to \infty} \mathrm{Prob}(D \textit{ is smooth}) = \prod_{P \in \IP^M} \mathrm{Prob}(D \textit{ is smooth at  }\pi^{-1}(P))$$  
where $D \in H^0(X, \sO(nA + dE))$ is a randomly chosen divisor.

The techniques developed in Poonen's paper \cite{PBertini} turns out to be surprisingly fruitful in that they have many useful by-products. For example, Poonen proved that over finite fields the classical Bertini smoothness theorem fails! Nyugen, who is a student of Poonen, proved an analogue of Whitney embedding theorems in his PhD thesis \cite{Nyugen}. 

\paragraph{Previous Work}
Over the summer I worked on some problems in a bigraded setting suggested by Prof. Schoen. I first proved the following
\begin{theorem}
Suppose $p = \mathrm{char}\, \IF_q \neq 2$. The aymptotic probability as $d \to \infty$ for a curve in $\IP^1 \times \IP^1$ over $\IF_q$ of bidegree $(3, d)$ to be simply ramified with respect to the projection to the second component is $\zeta_{\IP_{\IF_q}}(2)^{-2}$. To be precise, let $X = \IP^1 \times \IP^1$, $A = \sO_X(1, 1)$, $E = \sO_X(0, 1)$, $n = 3$ and $R_{n,d} = H^0(X, \sO_X(nA + d E))$. Then 
$$\lim_{d \to \infty} \frac{|f \in R_{n, d} : H_f \textit{ is simply ramified}|}{|R_{n,d}|} = \zeta_{\IP^1_{\IF_q}}(2)^{-2} $$ In particular, the conditional probability for a randomly chosen smooth curve to be simply ramified is $\zeta_{\IP^1_{\IF_q}}(2)/\zeta_{\IP^1_{\IF_q}}(3)$. 
\end{theorem}
In proving the result we first extended Poonen's decoupling idea to treat second order partial derivatives. Then I worked on an analogue of the above problem and proved the following 
\begin{theorem}
Suppose $p = \mathrm{char}\, \IF_q \neq 2, 3$. Let $X = \IP^2 \times \IP^1$, $R_{3,d} = H^0(X, \sO_X(3, d))$. Let $\pi : X \to \IP^1$ be the projection to second component. Then 
$$\lim_{d \to \infty} \frac{|f \in R_{3, d} : \textit{singular fibers of } H_f \textit{ w.r.t. $\pi$ are all nodal curves}|}{|R_{3,d}|} > 0 $$ 
In particular, when $d$ is large enough, then there always exists a hypersurface of bidegree $(3, d)$ whose singular fibers over $\IP^1$ are either smooth or nodal curves. 
\end{theorem}
The above result is harder than the previous one in that we need to sieve out those hypersurfaces with singular fibers that are reducible, but irreducibility is not in general a local property like smoothness. To show the above we used a complete Taylor expansion to analyze the tangent cones at points in order to sieve out reducible curves and cuspidal curves. 

\paragraph{Directions for current work} I am interested in how we can use sieve methods to systematically treat global properties that cannot be tested locally, for example, irreducibility. I know Prof. Poonen has already written an article on this \cite{Pirred}, introducing new ideas that I have not yet absorbed. Probably I can write something on the semiample version of his Bertini irreducibility theorems.

At the end of his expository aritle, Poonen posed the following questions:
\begin{enumerate}
\item There seems to be a general principle that if an existence result about polynomials
or n-tuples of polynomials over an infinite field can be proved by dimension counting,
then a corresponding result over finite fields can be proved by the closed point sieve.
Can this principle be formalized and proved?

\item What other theorems currently require the hypothesis “Assume that k is an infinite
field”? Hopefully the closed point sieve could be used to eliminate the hypothesis in
many of these.
\end{enumerate}

My research is basically following this theme. We treated the simply ramified curves and elliptic surface by first demonstrating a dimension counting argument and then converting the argument into a sieve method. An interesting phenomenon that I noticed is that sometimes there are more than one natural ways to do the dimension counting argument, but not all of them can be easily converted to a sieve argument. 

Prof. Schoen suggests that I read about ruled surfaces and see how my results for curves $\IP^1 \times \IP^1$ extend. 

\paragraph{Techniques} The main technique in the works mentioned above is sieve method, which is often tailored in various ways for different purposes. The arguments invariably involve interpolation, Lang-Weil bound, Bezout's theorem and Poonen's decoupling trick. Poonen's work on Bertini irreducibility theorems also introduce new ingredients: resolution of singularities for surfaces, cones of curves in a surface, and the function field Chebotarev density theorem. 

\section{Nature of Final Product}

A formal final paper (as opposed to an informal report) meeting the following criteria will be written in this course:

The paper describes some important aspects of the work done during the course.
The paper is thoughtful, well organized, well written and carefully proofread.
The paper communicates well to as broad an audience as would reasonably be assumed to be interested in the topic of the research.
The final paper will contribute substantially to the course grade. In particular an A grade will not be given if the final paper clearly fails to meet the criteria above.

The final paper may at the discretion of the instructor include relevant excerpts from any paper or report that the student has written for a previous related independent study course or in connection with a previous related research project. However such excerpts may only constitute a small portion of the final paper.

\section{Scheduled Meetings and Work Expectations}



\section{Grade Basis}
 The grade will be based on the final
paper as well as the instructor's assessment of progress.
Progress may involve either learning new material related
to the research or proving new results.


\begin{thebibliography}{9}
\bibitem{Pirred}
B. Poonen, \textit{Bertini irreducibility theroems over finite fields}, available at \textit{http://www-math.mit.edu/~poonen/papers/bertini\_irred.pdf}

\bibitem{Wood}
D. Erman and M.M. Wood, \textit{Semiample Bertini theorems over finite fields}, Duke Mathematical Journal 164(2015), no. 1, 1-38

\bibitem{PBertini}
B. Poonen, \textit{Bertini theorems over finite fields}, Ann. of Math. (2) 160 (2004), no. 3, 1099-1127.

\bibitem{Expo}
B. Poonen, \textit{Sieve methods for varieties over finite fields and arithmetic schemes}, J. Theor. Nombres Bordeaux, 19(1):221–229, 2007.

\bibitem{Nyugen}
N.H. Nguyen. \textit{Whitney theorems and Lefschetz pencils over finite fields}, 2005. Thesis (Ph.D.)–
University of California, Berkeley.
\end{thebibliography}





\end{document}