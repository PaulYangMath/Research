\documentclass[12pt]{article}
\usepackage{amsmath}
\usepackage{enumerate}
\usepackage{mathrsfs} 
\usepackage{amsthm}
\usepackage{amsfonts}
\usepackage{amssymb}
\usepackage{latexsym} 
%\usepackage{epsfig}
%\usepackage{graphicx}
%\usepackage[dvips]{graphicx}
\usepackage{tikz}
\usepackage{tikz-cd}



\usepackage[matrix,tips,graph,curve]{xy}

\newcommand{\mnote}[1]{${}^*$\marginpar{\footnotesize ${}^*$#1}}
\linespread{1.065}

\makeatletter

\setlength\@tempdima  {5.5in}
\addtolength\@tempdima {-\textwidth}
\addtolength\hoffset{-0.5\@tempdima}
\setlength{\textwidth}{5.5in}
\setlength{\textheight}{8.75in}
\addtolength\voffset{-0.625in}

\makeatother

\makeatletter 
\@addtoreset{equation}{section}
\makeatother


\renewcommand{\theequation}{\thesection.\arabic{equation}}

\theoremstyle{plain}
\newtheorem{theorem}[equation]{Theorem}
\newtheorem{corollary}[equation]{Corollary}
\newtheorem{conjecture}[equation]{Conjecture}
\newtheorem{lemma}[equation]{Lemma}
\newtheorem{proposition}[equation]{Proposition}
\theoremstyle{definition}
\newtheorem{definition}[equation]{Definition}
\newtheorem{definitions}[equation]{Definitions}
%\theoremstyle{remark}

\newtheorem{remark}[equation]{Remark}
\newtheorem{remarks}[equation]{Remarks}
\newtheorem{exercise}[equation]{Exercise}
\newtheorem{example}[equation]{Example}
\newtheorem{examples}[equation]{Examples}
\newtheorem{notation}[equation]{Notation}
\newtheorem{question}[equation]{Question}
\newtheorem{assumption}[equation]{Assumption}
\newtheorem*{claim}{Claim}
\newtheorem{answer}[equation]{Answer}
%%%%%% letters %%%%

\newcommand{\fa}{\mathfrak{a}}
\newcommand{\fb}{\mathfrak{b}}
\newcommand{\fm}{\mathfrak{m}}
\newcommand{\fp}{\mathfrak{p}}
\newcommand{\fq}{\mathfrak{q}}

\newcommand{\IA}{\mathbb{A}}
\newcommand{\IN}{\mathbb{N}}
\newcommand{\IF}{\mathbb{F}}
\newcommand{\IP}{\mathbb{P}}
\newcommand{\IZ}{\mathbb{Z}}

\newcommand{\sD}{\mathcal{D}}
\newcommand{\sI}{\mathcal{I}}
\newcommand{\sO}{\mathcal{O}}
\newcommand{\sP}{\mathcal{P}}
\newcommand{\sQ}{\mathcal{Q}}
\newcommand{\sT}{\mathcal{T}}
\newcommand{\sU}{\mathcal{U}}

\newcommand{\shF}{\mathscr{F}}
\newcommand{\shG}{\mathscr{G}}
%%%%%%% macros %%%%%

%% my definitions %%%

\newcommand{\End}{\mathrm{End}}
\newcommand{\tr}{\mathrm{tr}}
\newcommand{\Hom}{\mathrm{Hom}}
\newcommand{\Aut}{\mathrm{Aut}}
\newcommand{\Trace}{\mathrm{Trace}\,}
\newcommand{\rank}{\mathrm{rank}}
\renewcommand{\deg}{\mathrm{deg}\,}
\newcommand{\Spec}{\rm Spec\,}
\newcommand{\Proj}{\rm Proj\,}
\newcommand{\Sym}{\mathrm{Sym \,}}
\newcommand{\Span}{\mathrm{Span \,}}
\renewcommand\dim{{\rm dim\,}}
\newcommand{\codim}{{\rm codim\,}}
\renewcommand\det{{\rm det\,}}
\newcommand{\im}{{\rm Im\,}}


\newcommand\iso{{\, \simeq \,}} 
\newcommand\tensor{{\otimes}}
\newcommand\Tensor{{\bigotimes}} 
\newcommand\union{\bigcup} 
\newcommand\onehalf{\frac{1}{2}}
\newcommand\trivial{{\mathbb I}}
\newcommand\wb{\overline}

%%%%%Delimiters%%%%

\newcommand{\<}{\langle}
\renewcommand{\>}{\rangle}

%\renewcommand{\(}{\left(}
%\renewcommand{\)}{\right)}


%%%% Different kind of derivatives %%%%%

\newcommand{\delbar}{\bar{\partial}}
\newcommand{\pdu}{\frac{\partial}{\partial u}}
%\newcommand{\pd}[1][2]{\frac{\partial #1}{\partial #2}}

%%%%% Arrows %%%%%
\newcommand{\induce}{\rightsquigarrow}
\newcommand{\into}{\hookrightarrow}
\newcommand{\onto}{\twoheadrightarrow}
\newcommand{\tto}{\longmapsto}
\def\llra{\longleftrightarrow}
\def\wt{\widetilde}
\def\wtilde{\widetilde}
\def\what{\widehat}
\def\bf{\textbf}
\def\it{\textit}
%%%%%%%%%%%%%%%%%%% Ziquan's definitions %%%%%%%%%%%%%%%%%%%%
\newcommand{\Ann}{\mathrm{Ann}}
\newcommand{\height}{\mathrm{height \,}}
\newcommand{\Div}{\mathrm{Div}}
\newcommand{\sE}{\mathcal{E}}
\newcommand{\p}{\partial}
\newcommand{\Ohm}{\Omega}
\newcommand{\w}{\omega}
\newcommand{\sing}{\mathrm{sing}}
%%%%%%%%%%%%% new definitions for the positive mass paper %%%%%%%%%

\newcommand{\sperp}{{\scriptscriptstyle \perp}}
\newcommand{\Qmed}{\mathcal{Q}_r^\mathrm{medium}}
\newcommand{\Qhigh}{\mathcal{Q}^\mathrm{high}}
\newcommand{\uppermu}{\overline{\mu}}
\newcommand{\lowermu}{\underline{\mu}}
\newcommand{\res}{\mathrm{res}}
\newcommand{\ev}{\mathrm{ev}}
\newcommand{\pr}{\mathrm{pr}}
\newcommand{\Prob}{\mathrm{Prob}}
\newcommand{\st}{\, \mathrm{ s.t. }\,}
%%%%%%%%%%%%%%%%%%%%%%%

%%%%%%%%%%%%%%%%%%%%%%%%%%%%%%%%%%%%%%%%%%%%%



%
\begin{document}
%

\title{Simply Ramified Curves in $\IP^1 \times \IP^1$}
\author{Ziquan Yang}


\date{July 15th, 2015}

\maketitle

 
%\setcounter{secnumdepth}{1} 

\setcounter{section}{0}
\section{Introduction}
This note first presents an analogue of the traditional Bertini theorem that shows that almost all curves in $\IP^1 \times \IP^1$ are simply ramified with respect to projection to the second component, that is, such curves comprise a Zariski dense subset in some complete linear systems. Then we prove a Poonen type theorem for simply ramified curves of bidegree $(3, d)$ in $\IP^1 \times \IP^1$ over a finite field $\IF_q$. The outline of the proof is largely inspired by the Erman and Wood's work on semiample Bertini theorems over finite fields. However, we give an adaptation of Poonen's decoupling idea to treat second order partial derivatives.  
\section{An analogue of Bertini's theorem}
\begin{theorem}
Let $X = \IP^1 \times \IP^1$ and $V = \Gamma(X, \sO_X(n, d))$. Then $\IP V$ parametrizes curves of bidegree $(n, d)$. Let $\pi : X \to \IP^1$ be the projection to the second component. Let $P^0$ be the subscheme of $\IP V$ parametrizing smooth curves, then the subscheme $D$ parametrizing curves that are \textit{not} simply ramified with respect to $\pi$ is contained in a proper closed subscheme of $P^0$. 
\end{theorem} 
\begin{proof}
Given $f \in V$, let $H_f$ denote the curve in $\IP^2$ described by $f = 0$. Given $Q \in \IP^1$, we use $X_Q$ to denote the fiber over $Q$ under $\pi$, i.e. $\pi^{-1}(Q)$. We would use these conventions throughout this note. Roughly speaking, there are two types of curves in $D$. Type I curves are those that have a ramfication degree $\ge 3$ at some point. Type II curves are those that ramify at $\ge 2$ points along some fiber. For our purposes we may temporarily disregard the smoothness condition and define 
\begin{align*} 
D_1 &= \{ f \in \IP V : \exists P \in X, \st (H_f \cdot X_{\pi(P)})_P \ge 3 \} \\
D_2 &= \{ f \in \IP V : \exists Q \in \IP^1, \st \exists P_k \in X_Q, (H_f \cdot X_{Q})_{P_k} \ge 2, k = 0, 1 \}
\end{align*}
Clearly $D = P^0 \cap (D_1 \cup D_2)$ and hence it suffices to prove the following two claims. \\\\
\textit{Claim 1 :} $D_1$ is a proper closed subscheme of $\IP V$. \\
This is more or less a standard application of Bertini's arguments. We want $B \subseteq X \times \IP V$ to be the subsheme parametrizing pairs $(P, f)$ such that $(H_f \cdot X_{\pi(P)})_P \ge 3$. It suffices to describe $B$ locally. For $P$ we choose a standard affine chart $\IA^1 \times \IA^1$ containing it, and $f \in \IP V$ may be written as 
$$ f = \sum_{0 \le i \le n} \sum_{0 \le j \le d} a_{ij} x^i y^j $$
where $(a_{ij})$ can be thought of as coordinates of $\IP V$. Then the requirement $ f \in B_P $ can be described as 
$$ f(P) = \frac{ \p }{\p x } f (P) = \frac{ \p^2 }{\p x^2 } f(P) = 0 $$
This also shows that $B \cap (\IA^1 \times \IA^1) \times \IP V$ for each affine chart is closed, and hence $B$ is closed. For each $P$, $f \in B_P$ amounts to $3$ linear conditions on $\IP V$. Since $P$ takes values in $X$, and $\dim X = 2$, we have that $\dim B \le \IP V - 1$. $D_1$ is its projection to $\IP V$ and hence $D_1$ is closed and is of dimension $\le \IP V - 1$. \\\\
\textit{Claim 2 :} $D_2$ is a contained in a proper closed subscheme of $\IP V$. \\
The proof is not as straightforward as the previous one, since the question is no longer local at a point of $X$. Nonetheless we may sieve on pairs of points along the same fiber. Consider the space $ X \times X$. We label its coordinates as $((x_0 : x_1), (y_0 : y_1), (s_0 : s_1), (t_0 : t_1))$. Let $T'$ be the subscheme defined by $y_0 t_1 - y_1 t_0 = 0$ and let $\triangle$ be the diagonal of $ X \times X$, i.e. $\{y_0 t_1 - y_1 t_0 = 0, x_0 s_1 - x_1 s_0 = 0 \}$. Define $T = T' - \triangle$. Clearly $T$ is simply parametrizing pairs on $X$ that project to the same point under $\pi$ and we denote the projection of a pair of points in $T$ to their common image under $\pi$ by $\Pi$. This time we let $B$ be the subscheme of $ T \times \IP V$ parametrizing pairs $((P_0, P_1), f)$ such that $(H_f \cdot X_{Q})_{P_k} \ge 2,  k= 0, 1$. 

We observe that each $\IA^1 \times \IP^1 \cap T'$ can be covered by three copies of $\IA^1 \times \IA^1$, since it suffices to cover $\IP^1$ with three affine charts $\IA^1$ such that each pair is contained in one of them. For each pair $(P_0, P_1) \in T$, the requirement that $ f \in (T \times \IP V)_{(P_0, P_1)} $ amounts to 
$$ f(P_0) = \frac{ \p }{\p x } f (P_0) = 0 \text{ and } f(P_1) = \frac{ \p }{\p x } f (P_1) = 0$$ i.e. $4$ linear conditions on $\IP V$. Note that $\dim T = 3$. By a similar argument as in the proof of claim 1 we see that again $\dim B = \IP V - 1$ and $D_2$ is precisely the image of $B$ under the projection to $\IP V$. 

\end{proof}

\section{A density result}
\begin{theorem}
The aymptotic probability as $d \to \infty$ for a curve in $\IP^1 \times \IP^1$ over $\IF_q$ of bidegree $(3, d)$ to be simply ramified with respect to the projection to the second component is $\zeta_{\IP_{\IF_q}}(2)^{-2}$. To be precise, let $X = \IP^1 \times \IP^1$, $A = \sO_X(1, 1)$, $E = \sO_X(0, 1)$, $n = 3$ and $R_{n,d} = H^0(X, \sO_X(nA + d E))$. Suppose $p = \mathrm{char } \IF_q \neq 2$. Then 
$$\lim_{d \to \infty} \frac{|f \in R_{n, d} : H_f \textit{ is simply ramified}|}{|R_{n,d}|} = \zeta_{\IP_{\IF_q}}(2)^{-2} $$ In particular, the conditional probability for a randomly chosen smooth curve to be simply ramified is $\zeta_{\IP_{\IF_q}}(2)/\zeta_{\IP_{\IF_q}}(3)$. 
\end{theorem}
In the setting of Erman and Wood's paper, we have $$\iota : X \stackrel{|A|}{\to} \IP^3, \pi : X \stackrel{|E|}{\to} \IP^1$$ $\iota$ in our case is simply the Segre embedding and $\pi$ is projection to the second component. We say $f \in R_{n, d}$ is ``simply ramified" (resp. ``multiply ramified") at a point $Q \in X$ if $H_f$ and $X_{\pi(Q)}$ intersect with multiplicity $2$ (resp. $\ge 3$) at $Q$. We say $f \in R_{n, d}$ is ``good" at $Q \in X$, if $H_f$ is both smooth and simply ramified at $Q$, and ``bad" otherwise. For a fixed $e_0 \in \IN$, we define 
\begin{align*}
\sP_{e_0}^{\mathrm{low}} &= \union_{d \ge 0} \{ f \in R_{n, d} : f \textit{ is good at all $Q \in X$,}\deg \pi(Q) < e_0\}\\
\sQ_{e_0}^{\mathrm{med}} &= \union_{d \ge 0} \{f \in R_{n, d} : f \textit{ is bad at some $Q$, }\deg \pi(Q) \in [e_0, \lfloor d/p \rfloor]\}\\
\sQ^{\mathrm{high}} &= \union_{d \ge 0} \{f \in R_{n, d} : f \textit{ is bad at some $Q$, }\deg \pi(Q) > d/p\}
\end{align*}
In accordance with the convention in Erman and Wood's paper, by ``probability" we often mean the asymptotic probability as $d \to \infty$. To be precise, if $\sP \subseteq \union_d R_{n, d}$, we define 
$$ \Prob(f \in \sP) = \lim_{d \to \infty}\Prob(f_d \in \sP) $$
where $f$ and $f_d$ are randomly chosen from $\union_{d} R_{n, d}$ and $R_{n, d}$ respectively. We first compute the probability that $f$ is good at a single fiber:
\begin{lemma} 
\label{count}
Given a closed point $P \in \IP^1$ with $\deg P = e$. The probability that $f \in R_{n, d}$ is good at all points on the fiber $X_P$ is 
$$ ( 1 - q^{- 2e})^{-2} $$ for $d$ large enough. 
\end{lemma}
\begin{proof}
I will fill in this proof later. A nicer argument than what I currently have could be given once I understand Erman and Wood's counting method. I shall also describe what is ``large enough". 
\end{proof}
We prove the theorem by proving three lemmas:
\begin{lemma}
$$\Prob(f \in \sP_{e_0}^{\mathrm{low}}) = \prod_{\deg(P) < e_0} \Prob(H_f \textit{ is good at all points of } \pi^{-1}(P))$$
\end{lemma}
\begin{proof}
Essentially the same proof as that of lemma 3.2 in \cite{Wood}. 
\end{proof}
\begin{lemma}
$$\lim_{e_0 \to \infty} \Prob(f \in \sP_{e_0}^{\mathrm{med}}) = 0$$
\end{lemma}
\begin{proof}
For $\deg P \le \lfloor d/p \rfloor$, $d$ is large enough to apply lemma~\ref{count}. Therefore we may compute 
\begin{align*}
\Prob( f \in \sP_{e_0}^{\mathrm{med}} ) &\le \lim_{d \to \infty} \sum_{e = e_0}^{ \lfloor d/p \rfloor } (2 q^{-2e} - q^{-4e}) q^e \\
&\le \sum_{e = e_0}^{\infty} 2q^{-e}
\end{align*}
Of course, as $e_0 \to \infty$, it goes to zero. 
\end{proof}
\begin{lemma}
$$\Prob( f \in \sQ^{\mathrm{high}}) = 0 $$
\end{lemma}
\begin{proof}
As in the proof of the analogue of Bertini theorem, $\sQ^{\mathrm{high}} = \sQ^{\mathrm{high}}_{\mathrm{sing}} \cup \sQ^{\mathrm{high}}_{\mathrm{multi}}$. As their name suggests, $\sQ^{\mathrm{high}}_{\mathrm{sing}}$ are those curves that are bad for being singular and $\sQ^{\mathrm{high}}_{\mathrm{multi}}$ for being multiply ramified. It is already shown in \cite{Wood} that $$ \Prob(f \in \sQ^{\mathrm{high}}_{\mathrm{sing}}) = 0 $$ and hence we only need to show
$$ \Prob(f \in \sQ^{\mathrm{high}}_{\mathrm{multi}}) = 0 $$
The proof is completed by taking $d \to \infty$ in the following lemma. 
\end{proof}

\begin{lemma}
Let $f \in R_{n, d}$ be chosen and random and $j \in \IN$ be fixed. The probability $P^*$ that $H_f$ has a multiply ramified point $Q \in X$ with $\deg \pi(Q) \ge j$ is at most
$$ \sO((n + d)q^{- \min\{ \lfloor d/p \rfloor + 1, j    \}})$$
\end{lemma}
For the spaces involved in morphisms $\iota : X \to \IP^3$ and $\pi : X \to \IP^1$ we label the coordinates of $X$ as $((x_0 : x_1), (y_0 : y_1))$, those of $\IP^3$ as $(s_0 : s_1 : s_2 : s_3)$ and those of $\IP^1$ as $(t_0 : t_1)$. We only consider their affine chart on which the first coordinate does not vanish and may use the same letters to denote the dehomogenized coordinates, but we will use ``," instead of ``:" for affine coordiantes. In the proof we use $\iota$ to identify $X$ with a subscheme of $\IP^3$. On affine chart $\IA^2 = \{ x_0, y_0 \neq 0 \} \subseteq X$, $\iota$ restricts to $$(x, y) \mapsto (x, y, xy) \in \IA^3 = \{s_0 \neq 0\} \subseteq \IP^3$$. We can cover $X$ by finitely many affine opens, so we may restrict to $U = X \cap \IA^3$. On $U$ we see that $s_3 = s_1 s_2$ and hence $ds_3 = s_2 ds_1 + s_1 ds_2$. Therefore the differentials in $U$ are generated by $ds_1, ds_2$. For each $f \in A = \IF_q[s_1, s_2, s_3]$, on $U$ $df$ has a unique decomposition $df = D_1 f ds_1 + D_2 f ds_2$. This gives maps $D_i : A \to A$, which we should think of as derivatives of pullbacks. To be precise, if $\Phi : A \to \IF_q[x, y]$ is the pullback map of regular functions, then $$\Phi(D_1 f) = \frac{\p}{\p x} \Phi(f), \text{ and }\Phi(D_2 f) = \frac{\p}{\p y} \Phi(f)$$ As an example, let $f = s_3$, then $D_1 f = s_2, D_2 f = s_1$, and the verification is straightforward. 

Now let me explain a simple linear algebraic observation. Let $V$ be a $n$-dimensional vector space over $\IF_q$ and $\ker L$ is a $k$-dimensional subspace ($L$ is some linear map). Let $W \subseteq V$ be a $m$-dimensional subspace, then $\dim W \cap \ker L \ge m + k - n$, and hence $$\frac{|W \cap \ker L|}{|W|} \ge \frac{q^{m + k - n}}{q^m} = q^{k - n} = \frac{|\ker L|}{|V|}$$ Note that the image of a curve in the second infinitestimal neighborhood $Q^{(3)}$ on $U$ already contains enough informaiton to decide whether the curve is multiply ramified at $Q$ and the map $S_{n, d} = H^0(\IP^3 \times \IP^1, \sO(n, d)) \to H^0(Q^{(3)}, \sO_{Q^{(3)}})$ factors through $\phi : S_{n, d} \to R_{n, d}$. The probability that a randomly chosen $f \in R_{n, d}$ to give a curve multiply ramified at $Q$ is bounded by that of $g \in S_{n, d}$. Therefore we reduce to studying a randomly chosen $g \in S_{n, d}$.  

Locally $\phi$ is viewed as follows: On chart $\IA^1 = \{t_0 \neq 0 \} \subseteq \IP^1$, the dehomogenized coordinate function $t = t_1/t_0$ pulls back via $\pi$ to a regular function on $U$, but in practice it only means on $U$ we identify $t$ with $y$ or $s_2$. Recall that for a point $Q \in U$, $g \in S_{n, d}$ is multiply ramified at $Q$ if and only if $$Q \in \{ D_1 \phi(g) = D_1^2 \phi(g) = \phi(g) = 0 \}$$ Since we are primarily concerned with $D_1$ we would drop the supscript and write $D_1 = D$ and similarly $s = s_1$. 

We apply the Poonen's decoupling idea to decouple the first and second order partial derivatives in this context. To choose $g \in S_{n, d}$ at randomly we choose $g_0 \in S_d$, $g_1, g_2, h \in S_{0, \lfloor d/p \rfloor} $ at random and put 
$$ g = g_0 + g_1^p s^2 + g_2^p s + h^p $$ This has the advantage that 
$$ D \phi(g) = D \phi(g_0) + 2 \phi(g_1^p s) + \phi(g_2)^p \text{ and } D^2 \phi(g) = D^2 \phi(g_0) + 2 \phi(g_1)^p $$
Define $W_2 = U \cap \{ D^2 \phi(g) = 0 \}$, $W_1 = U \cap \{D^2 \phi(g) = D \phi(g) = 0\}$ and $W_0 = U \cap \{ D^2 \phi(g) = D \phi(g) = \phi(g) = 0\}$. \\\\
\textit{Claim: }Conditioned on a choice of $g_0$, the probability that $\dim W_2 = 1$ and $W_2$ contains a point $Q$ with $\deg \pi(Q) \ge j$ (following Erman and Wood's convention we call such $Q$ an ``admissible" point) is $$1 - \sO((n + d)q^{- \min\{ \lfloor d/p \rfloor + 1, j\}})$$ Let $V_1, \cdots, V_c$ be irreducible components of $\{D^2(\phi(g_0)) = 0\}_{\mathrm{red}}$. For each $V_k$ that contains an admissible point, those $g_1$ that will make $D^2 \phi(g)$ vanish identically on $V_k$ form a coset of the linear map $$ \varphi_k : S_{0, \lfloor n/p \rfloor} \to H^0(V_k, \sO_{V_k})$$ A lower bound on $|\im \varphi_k|$ gives an upper bound for the probability that $D^2(g)$ vanishes on $V_k$, which is simply the inverse of $|\im \varphi_k|$. If $\dim \pi(V_k) = 1$, we see that the pullback of $t$, and hence any polynomial in $t$, does not vanish on $V_k$. In this case, we have $|\im \varphi_k| \ge q^{\lfloor d/p \rfloor + 1}$. If $\dim \pi(V_k) = 0$, we invoke lemma 5.3 in \cite{Wood} to obtain $|\im \varphi_k| \ge q^{\min \{\lfloor d/p \rfloor + 1, j \}  }$. The claim hence follows. Similarly, we can show that the probability that $\dim W_1 = 0$ conditioned on a choice of $g_0, g_1$ such that $\dim W_2 = 1$, and the probability that $W_0 \cap W_1$ contains no admissible points conditioned on a choice that $\dim W_1 = 0$ are both bounded by $$1 - \sO((n + d)q^{- \min\{ \lfloor d/p \rfloor + 1, j\}})$$
Therefore we have shown that 
$$ P^* \le 1 - (1 - \sO((n + d)q^{- \min\{ \lfloor d/p \rfloor + 1, j\}}))^3 = \sO((n + d)q^{- \min\{ \lfloor d/p \rfloor + 1, j\}}) $$ 

\section{Conclusion}
For curves of bidegree $(3, d)$ we have the luxury of dealing only with those type I curves as in the analogue of Bertini theorem that we proved earlier. I conjecture that the independence across fibers still holds for $n \ge 3$. As our earlier proof suggests, this may involve a sieve on pairs of points. It seems that the paper by Poonen's student, Nyugen gives an example of treating pairs of points. Hence I will see if I can borrow some ideas in his paper to treat the general case $n \ge 3$. 

\begin{thebibliography}{9}
\bibitem{Wood} 
D. Erman and M.M. Wood, \textit{Semiample Bertini theorems over finite fields}, Duke Mathematical Journal 164 (2015), no. 1, 1-38
\end{thebibliography}

\end{document}
