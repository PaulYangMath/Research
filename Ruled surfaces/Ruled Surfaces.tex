\documentclass[12pt]{article}
\usepackage{amsmath}
\usepackage{enumerate}
\usepackage{mathrsfs} 
\usepackage{amsthm}
\usepackage{amsfonts}
\usepackage{amssymb}
\usepackage{latexsym} 
%\usepackage{epsfig}
%\usepackage{graphicx}
%\usepackage[dvips]{graphicx}
\usepackage{tikz}
\usepackage{tikz-cd}



\usepackage[matrix,tips,graph,curve]{xy}

\newcommand{\mnote}[1]{${}^*$\marginpar{\footnotesize ${}^*$#1}}
\linespread{1.065}

\makeatletter

\setlength\@tempdima  {5.5in}
\addtolength\@tempdima {-\textwidth}
\addtolength\hoffset{-0.5\@tempdima}
\setlength{\textwidth}{5.5in}
\setlength{\textheight}{8.75in}
\addtolength\voffset{-0.625in}

\makeatother

\makeatletter 
\@addtoreset{equation}{section}
\makeatother


\renewcommand{\theequation}{\thesection.\arabic{equation}}

\theoremstyle{plain}
\newtheorem{theorem}[equation]{Theorem}
\newtheorem{corollary}[equation]{Corollary}
\newtheorem{conjecture}[equation]{Conjecture}
\newtheorem{lemma}[equation]{Lemma}
\newtheorem{proposition}[equation]{Proposition}
\theoremstyle{definition}
\newtheorem{definition}[equation]{Definition}
\newtheorem{definitions}[equation]{Definitions}
%\theoremstyle{remark}

\newtheorem{remark}[equation]{Remark}
\newtheorem{remarks}[equation]{Remarks}
\newtheorem{exercise}[equation]{Exercise}
\newtheorem{example}[equation]{Example}
\newtheorem{examples}[equation]{Examples}
\newtheorem{notation}[equation]{Notation}
\newtheorem{question}[equation]{Question}
\newtheorem{assumption}[equation]{Assumption}
\newtheorem*{claim}{Claim}
\newtheorem{answer}[equation]{Answer}
%%%%%% letters %%%%

\newcommand{\fa}{\mathfrak{a}}
\newcommand{\fb}{\mathfrak{b}}
\newcommand{\fm}{\mathfrak{m}}
\newcommand{\fp}{\mathfrak{p}}
\newcommand{\fq}{\mathfrak{q}}

\newcommand{\IA}{\mathbb{A}}
\newcommand{\IN}{\mathbb{N}}
\newcommand{\IF}{\mathbb{F}}
\newcommand{\IP}{\mathbb{P}}
\newcommand{\IZ}{\mathbb{Z}}

\newcommand{\sD}{\mathcal{D}}
\newcommand{\sI}{\mathcal{I}}
\newcommand{\sO}{\mathcal{O}}
\newcommand{\sP}{\mathcal{P}}
\newcommand{\sQ}{\mathcal{Q}}
\newcommand{\sT}{\mathcal{T}}
\newcommand{\sU}{\mathcal{U}}

\newcommand{\shF}{\mathscr{F}}
\newcommand{\shG}{\mathscr{G}}
%%%%%%% macros %%%%%

%% my definitions %%%

\newcommand{\End}{\mathrm{End}}
\newcommand{\tr}{\mathrm{tr}}
\newcommand{\Hom}{\mathrm{Hom}}
\newcommand{\Aut}{\mathrm{Aut}}
\newcommand{\Trace}{\mathrm{Trace}\,}
\newcommand{\rank}{\mathrm{rank}}
\renewcommand{\deg}{\mathrm{deg}\,}
\newcommand{\Spec}{\rm Spec\,}
\newcommand{\Proj}{\rm Proj\,}
\newcommand{\Sym}{\mathrm{Sym \,}}
\newcommand{\Span}{\mathrm{Span \,}}
\renewcommand\dim{{\rm dim\,}}
\newcommand{\codim}{{\rm codim\,}}
\renewcommand\det{{\rm det\,}}
\newcommand{\im}{{\rm Im\,}}


\newcommand\iso{{\, \simeq \,}} 
\newcommand\tensor{{\otimes}}
\newcommand\Tensor{{\bigotimes}} 
\newcommand\union{\bigcup} 
\newcommand\onehalf{\frac{1}{2}}
\newcommand\trivial{{\mathbb I}}
\newcommand\wb{\overline}

%%%%%Delimiters%%%%

\newcommand{\<}{\langle}
\renewcommand{\>}{\rangle}

%\renewcommand{\(}{\left(}
%\renewcommand{\)}{\right)}


%%%% Different kind of derivatives %%%%%

\newcommand{\delbar}{\bar{\partial}}
\newcommand{\pdu}{\frac{\partial}{\partial u}}
%\newcommand{\pd}[1][2]{\frac{\partial #1}{\partial #2}}

%%%%% Arrows %%%%%
\newcommand{\induce}{\rightsquigarrow}
\newcommand{\into}{\hookrightarrow}
\newcommand{\onto}{\twoheadrightarrow}
\newcommand{\tto}{\longmapsto}
\def\llra{\longleftrightarrow}
\def\wt{\widetilde}
\def\wtilde{\widetilde}
\def\what{\widehat}
\def\bf{\textbf}
\def\it{\textit}
%%%%%%%%%%%%%%%%%%% Ziquan's definitions %%%%%%%%%%%%%%%%%%%%
\newcommand{\Ann}{\mathrm{Ann}}
\newcommand{\height}{\mathrm{height \,}}
\newcommand{\Div}{\mathrm{Div}}
\newcommand{\sE}{\mathcal{E}}
\newcommand{\p}{\partial}
\newcommand{\Ohm}{\Omega}
\newcommand{\w}{\omega}
\newcommand{\sing}{\mathrm{sing}}
%%%%%%%%%%%%% new definitions for the positive mass paper %%%%%%%%%

\newcommand{\sperp}{{\scriptscriptstyle \perp}}
\newcommand{\Qmed}{\mathcal{Q}_r^\mathrm{medium}}
\newcommand{\Qhigh}{\mathcal{Q}^\mathrm{high}}
\newcommand{\uppermu}{\overline{\mu}}
\newcommand{\lowermu}{\underline{\mu}}
\newcommand{\res}{\mathrm{res}}
\newcommand{\ev}{\mathrm{ev}}
\newcommand{\pr}{\mathrm{pr}}
\newcommand{\Prob}{\mathrm{Prob}}
%%%%%%%%%%%%%%%%%%%%%%%

%%%%%%%%%%%%%%%%%%%%%%%%%%%%%%%%%%%%%%%%%%%%%



%
\begin{document}
%

\title{Simply Ramified Curves on Ruled Surfaces}
\author{Ziquan Yang}


\date{\today}

\maketitle

 
%\setcounter{secnumdepth}{1} 

\setcounter{section}{0}

% After I sat down and started to think about the problem on a generic ruled surfaces, I soon realized that the first thing I need is a parametrization of curves. 

% For $\IP^1 \times \IP^1$ we can talk about bidegree. Without resorting to bidegrees, we can set $A = \sO_{\IP^1 \times \IP^1} (1, 0)$ and $E = \sO_{\IP^1 \times \IP^1} (0, 1)$ so bidegree $(n, d)$ is the same as $nA + dE$. Since we don't have a notion of bidegree on a ruled surface (or do we?) in general, I would like to find appropriate divisors. Let $\pi : X \to C$ be a ruled surface, I can find a divisor $A$ on $X$ such that $A \cdot X_y = 1$, where $X_y$ is a fiber. As Hartschorne has remarked, $A \cdot X_{y'} = 1$ if we replace $X_y$ by $X_{y'}$. The hard part is to find the other divisor that assumes the role of $E$. I am thinking of making use of a section $\sigma : C \to X$. I would like to find a divisor $E$ such that $E \cdot X_y = 0$ for a fiber $X_y$, and $E \cdot \sigma(C) = 1$. Then I can talk about curves given by $f \in H^0(X, \sO(nA + dE))$, which I think is the analogue of ``bidegree $(n,d)$". 

% I would still like to think of sections in $H^0(X, \sO(nA + dE))$ locally as ``polynomials of bidegree $(n, d)$". That means, if $y \in C$ is a point and $U \iso \IA^1 \subseteq C$ is an affine neighborhood containing the point, then the restriction of $f$ on $\pi^{-1}(U)$ is of the form $\sum p(x_0, x_1) q(t)$, $\deg p = n, \deg q = d$, where $((x_0, x_1), t)$ is the coordiante we give to $\pi^{-1}(U) \iso \IP^1 \times \IA^1$. 

% What do you think is the correct way of formulating it? 

For $X = \IP^1 \times \IP^1$ over $\IF_q$ we proved the following:
\begin{theorem}
Let $X = \IP^1 \times \IP^1$, $A = \sO_X(1, 0)$, $E = \sO_X(0, 1)$, $n = 3$ and $R_{n,d} = H^0(X, \sO_X(nA + d E))$. Suppose $p = \mathrm{char } \IF_q \neq 2$. Let $\pi : X \to \IP^1$ be the projection to the second component. Then 
$$\lim_{d \to \infty} \frac{|f \in R_{n, d} : H_f \textit{ is simply ramified w.r.t. }\pi|}{|R_{n,d}|} = \zeta_{\IP_{\IF_q}}(2)^{-2} $$ 
\end{theorem}

We would like to extend the result to ruled surfaces. The first thing that we need to do is to find an appropriate set $R_{n, d}$ of sections, so that we may end up with something like:

\begin{theorem}
Let $\pi : X \to C$ be a ruled surface. 
$$\lim_{d \to \infty} \frac{|f \in R_{n, d} : H_f \textit{ is simply ramified w.r.t} \pi|}{|R_{n,d}|} = \zeta_{\IP_{\IF_q}}(2)^{-2} $$
\end{theorem}





\end{document}